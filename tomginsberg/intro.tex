Machine learning models operate on the assumption, albeit incorrectly that they will be deployed on data distributed identically to what they were trained on.
The violation of this assumption is known as distribution shift and can often result in significant degradation of performance~\citep{cifar101,cifar10C,learnundercov, failloud, mindperfgap, trustuncert}.
There are several cases where a mismatch between training and deployment data results in very real consequences on human beings.
In healthcare, machine learning models have been deployed for predicting the likelihood of sepsis.
Yet, as~\citep{10.1001/jamainternmed.2021.3333} show, such models can be miscalibrated for large groups of individuals, directly affecting the quality of care they experience.
The deployment of classifiers in the criminal justice system~\citep{AIissend45:online}, hiring and recruitment pipelines~\citep{Amazonsc17:online} and self-driving cars~\citep{selfdrive} have all seen humans affected by the failures of learning models.
The need for methods that quickly detect, characterize and respond to distribution shift is, therefore, a fundamental problem in trustworthy machine learning.
For practitioners, regulatory agencies and individuals to have faith in deployed predictive models without the need for laborious manual audits, we need methods for the identification of distribution shift
that are \textit{sample-efficient} (identifying shifts from a small number of samples), \textit{informed} (identifying shifts relevant to the domain and learning algorithm), \textit{model-agnostic}
(identifying shifts regardless of the functional class of the predictive model) and \textit{statistically sound} (identifying true shifts while avoiding false positives with high-confidence).

In our work, we study a special case of distribution shift, commonly known as \textit{covariate shift}, which considers shifts only in the distribution of input data $\Pc_X \neq \Qc_X$ while the relation between the inputs and outputs remains fixed $\Pc_{Y|X} = \Qc_{Y|X}$.
In a standard deployment setting where ground truth labels are not available, estimating any properties of the true distribution $\Qc_{Y|X}$ is fundamentally impossible without strong assumptions.
As a consequence covariate shift is the only type of distribution shift that can be identified.

We build off recent progress in understanding model performance under covariate shift using the PQ-learning framework~\citep{pqlearn}, a framework for selective classifiers that may either predict on or reject a given sample, that provides strong performance guarantees on arbitrary test distributions.
Our work uses and extends PQ-learning to develop a practical, model-based hypothesis test, named \textit{the \method}, to identify potentially harmful covariate shifts given any existing classification model already in deployment.

Our work makes the following key contributions:

\begin{itemize}
    \item We show how to construct an ensemble of classifiers that maximize out-of-domain disagreement while behaving consistently in the training domain.
    We propose the \emph{disagreement cross entropy} for models learned via continuous gradient-based methods (e.g., neural networks), as well as a generalization for those learned via discrete optimization (e.g., random forest).
    \item We show that the rejection rate and the entropy of the learning ensemble can be used to define a model-aware hypothesis test for covariate shift, the \method,\ that in idealized settings can provably detect covariate shift.
    \item On high-dimensional image and tabular data, using both neural networks and gradient boosted decision trees, our method outperforms state-of-the-art techniques for detecting covariate shift, particularly when given access to as few as ten test examples.
\end{itemize}

\begin{figure}[!htb]
    \centering
    \includegraphics[width=\linewidth]{images/detectron.pdf}
    \caption{\small \textbf{Detectron (PQ Learning for Covariate Shift Detection):} Starting with a base classifier trained on labeled samples from distribution $\Pc$ we
    train new \textit{\textbf{C}onstrained \textbf{D}isagreement \textbf{C}lassifiers} (CDCs) on a small set of observed unlabeled samples from a new distribution $\Qc$.
    CDCs aim to maximize classification disagreement on $\Qc$ while constrained to agree with the base classifier on $\Pc$. The rate $\phi$ that CDCs disagree,
        as well as their entropy, on test data is a powerful and sample efficient statistic for identifying covariate shift $\Pc\neq \Qc$.}
    \label{fig:detectron}
\end{figure}