\input{math_commands.tex}
% packages
\usepackage[colorlinks]{hyperref} % for links
%\usepackage[backend=biber]{biblatex} % for references
%\addbibresource{bib.bib} % for references

\usepackage{graphicx} % for embedding graphics
\usepackage{booktabs} % for pretty tables
\usepackage{lipsum} % for gibberish text
\usepackage{mathtools}
\usepackage{wrapfig}
\usepackage{natbib}
\usepackage{floatrow}
\bibliographystyle{abbrvnat}
\setcitestyle{authoryear, citesep={;}, aysep={,},yysep={;}}
\setcitestyle{square}


\doublespacing
\usepackage{xcolor}
\definecolor{amaranth}{HTML}{E83151}
\definecolor{russianviolet}{HTML}{330036}
\definecolor{darkblue}{HTML}{645E9D}
\hypersetup{
    colorlinks=true,
    linkcolor=russianviolet,
    citecolor=darkblue,
    urlcolor=amaranth}
\newcommand{\xhdr}[1]{\noindent{{\bf #1.}}}
\usepackage{listings}
\usepackage{pythonhighlight}
\usepackage{makecell}
\usepackage{tabularx}

\usepackage{cleveref}
\usepackage{amssymb}
\usepackage{amsthm}

\usepackage[ruled,vlined]{algorithm2e}
\newcommand\mycommfont[1]{\footnotesize\ttfamily\textcolor{blue}{#1}}
\SetCommentSty{mycommfont}

\newcommand{\boldQ}{\mathbf{Q}}
\newcommand{\boldP}{\mathbf{P}}
\newcommand{\Qc}{\mathcal{Q}}
\newcommand{\Pc}{\mathcal{P}}
\newcommand{\Rc}{\mathcal{R}}
\newcommand{\Pun}{\boldP^\star}

\newcommand{\method}{Detectron}

\theoremstyle{definition}
\newtheorem{definition}{Definition}
\newtheorem{assumption}{Assumption}
\newtheorem{theorem}{Theorem}
\newtheorem{lemma}{Lemma}

\newtheorem{manualtheoreminner}{Theorem}
\newenvironment{manualtheorem}[1]{%
    \renewcommand\themanualtheoreminner{#1}%
    \manualtheoreminner}
    {\endmanualtheoreminner}
