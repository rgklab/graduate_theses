\documentclass{ut-thesis}

%%%%% NEW MATH DEFINITIONS %%%%%

\usepackage{amsmath,amsfonts,bm}

% Mark sections of captions for referring to divisions of figures
\newcommand{\figleft}{{\em (Left)}}
\newcommand{\figcenter}{{\em (Center)}}
\newcommand{\figright}{{\em (Right)}}
\newcommand{\figtop}{{\em (Top)}}
\newcommand{\figbottom}{{\em (Bottom)}}
\newcommand{\captiona}{{\em (a)}}
\newcommand{\captionb}{{\em (b)}}
\newcommand{\captionc}{{\em (c)}}
\newcommand{\captiond}{{\em (d)}}

% Highlight a newly defined term
\newcommand{\newterm}[1]{{\bf #1}}


% Figure reference, lower-case.
\def\figref#1{figure~\ref{#1}}
% Figure reference, capital. For start of sentence
\def\Figref#1{Figure~\ref{#1}}
\def\twofigref#1#2{figures \ref{#1} and \ref{#2}}
\def\quadfigref#1#2#3#4{figures \ref{#1}, \ref{#2}, \ref{#3} and \ref{#4}}
% Section reference, lower-case.
\def\secref#1{section~\ref{#1}}
% Section reference, capital.
\def\Secref#1{Section~\ref{#1}}
% Reference to two sections.
\def\twosecrefs#1#2{sections \ref{#1} and \ref{#2}}
% Reference to three sections.
\def\secrefs#1#2#3{sections \ref{#1}, \ref{#2} and \ref{#3}}
% Reference to an equation, lower-case.
\def\eqref#1{equation~\ref{#1}}
% Reference to an equation, upper case
\def\Eqref#1{Equation~\ref{#1}}
% A raw reference to an equation---avoid using if possible
\def\plaineqref#1{\ref{#1}}
% Reference to a chapter, lower-case.
\def\chapref#1{chapter~\ref{#1}}
% Reference to an equation, upper case.
\def\Chapref#1{Chapter~\ref{#1}}
% Reference to a range of chapters
\def\rangechapref#1#2{chapters\ref{#1}--\ref{#2}}
% Reference to an algorithm, lower-case.
\def\algref#1{algorithm~\ref{#1}}
% Reference to an algorithm, upper case.
\def\Algref#1{Algorithm~\ref{#1}}
\def\twoalgref#1#2{algorithms \ref{#1} and \ref{#2}}
\def\Twoalgref#1#2{Algorithms \ref{#1} and \ref{#2}}
% Reference to a part, lower case
\def\partref#1{part~\ref{#1}}
% Reference to a part, upper case
\def\Partref#1{Part~\ref{#1}}
\def\twopartref#1#2{parts \ref{#1} and \ref{#2}}

\def\ceil#1{\lceil #1 \rceil}
\def\floor#1{\lfloor #1 \rfloor}
\def\1{\bm{1}}
\newcommand{\train}{\mathcal{D}}
\newcommand{\valid}{\mathcal{D_{\mathrm{valid}}}}
\newcommand{\test}{\mathcal{D_{\mathrm{test}}}}

\def\eps{{\epsilon}}


% Random variables
\def\reta{{\textnormal{$\eta$}}}
\def\ra{{\textnormal{a}}}
\def\rb{{\textnormal{b}}}
\def\rc{{\textnormal{c}}}
\def\rd{{\textnormal{d}}}
\def\re{{\textnormal{e}}}
\def\rf{{\textnormal{f}}}
\def\rg{{\textnormal{g}}}
\def\rh{{\textnormal{h}}}
\def\ri{{\textnormal{i}}}
\def\rj{{\textnormal{j}}}
\def\rk{{\textnormal{k}}}
\def\rl{{\textnormal{l}}}
% rm is already a command, just don't name any random variables m
\def\rn{{\textnormal{n}}}
\def\ro{{\textnormal{o}}}
\def\rp{{\textnormal{p}}}
\def\rq{{\textnormal{q}}}
\def\rr{{\textnormal{r}}}
\def\rs{{\textnormal{s}}}
\def\rt{{\textnormal{t}}}
\def\ru{{\textnormal{u}}}
\def\rv{{\textnormal{v}}}
\def\rw{{\textnormal{w}}}
\def\rx{{\textnormal{x}}}
\def\ry{{\textnormal{y}}}
\def\rz{{\textnormal{z}}}

% Random vectors
\def\rvepsilon{{\mathbf{\epsilon}}}
\def\rvtheta{{\mathbf{\theta}}}
\def\rva{{\mathbf{a}}}
\def\rvb{{\mathbf{b}}}
\def\rvc{{\mathbf{c}}}
\def\rvd{{\mathbf{d}}}
\def\rve{{\mathbf{e}}}
\def\rvf{{\mathbf{f}}}
\def\rvg{{\mathbf{g}}}
\def\rvh{{\mathbf{h}}}
\def\rvu{{\mathbf{i}}}
\def\rvj{{\mathbf{j}}}
\def\rvk{{\mathbf{k}}}
\def\rvl{{\mathbf{l}}}
\def\rvm{{\mathbf{m}}}
\def\rvn{{\mathbf{n}}}
\def\rvo{{\mathbf{o}}}
\def\rvp{{\mathbf{p}}}
\def\rvq{{\mathbf{q}}}
\def\rvr{{\mathbf{r}}}
\def\rvs{{\mathbf{s}}}
\def\rvt{{\mathbf{t}}}
\def\rvu{{\mathbf{u}}}
\def\rvv{{\mathbf{v}}}
\def\rvw{{\mathbf{w}}}
\def\rvx{{\mathbf{x}}}
\def\rvy{{\mathbf{y}}}
\def\rvz{{\mathbf{z}}}

% Elements of random vectors
\def\erva{{\textnormal{a}}}
\def\ervb{{\textnormal{b}}}
\def\ervc{{\textnormal{c}}}
\def\ervd{{\textnormal{d}}}
\def\erve{{\textnormal{e}}}
\def\ervf{{\textnormal{f}}}
\def\ervg{{\textnormal{g}}}
\def\ervh{{\textnormal{h}}}
\def\ervi{{\textnormal{i}}}
\def\ervj{{\textnormal{j}}}
\def\ervk{{\textnormal{k}}}
\def\ervl{{\textnormal{l}}}
\def\ervm{{\textnormal{m}}}
\def\ervn{{\textnormal{n}}}
\def\ervo{{\textnormal{o}}}
\def\ervp{{\textnormal{p}}}
\def\ervq{{\textnormal{q}}}
\def\ervr{{\textnormal{r}}}
\def\ervs{{\textnormal{s}}}
\def\ervt{{\textnormal{t}}}
\def\ervu{{\textnormal{u}}}
\def\ervv{{\textnormal{v}}}
\def\ervw{{\textnormal{w}}}
\def\ervx{{\textnormal{x}}}
\def\ervy{{\textnormal{y}}}
\def\ervz{{\textnormal{z}}}

% Random matrices
\def\rmA{{\mathbf{A}}}
\def\rmB{{\mathbf{B}}}
\def\rmC{{\mathbf{C}}}
\def\rmD{{\mathbf{D}}}
\def\rmE{{\mathbf{E}}}
\def\rmF{{\mathbf{F}}}
\def\rmG{{\mathbf{G}}}
\def\rmH{{\mathbf{H}}}
\def\rmI{{\mathbf{I}}}
\def\rmJ{{\mathbf{J}}}
\def\rmK{{\mathbf{K}}}
\def\rmL{{\mathbf{L}}}
\def\rmM{{\mathbf{M}}}
\def\rmN{{\mathbf{N}}}
\def\rmO{{\mathbf{O}}}
\def\rmP{{\mathbf{P}}}
\def\rmQ{{\mathbf{Q}}}
\def\rmR{{\mathbf{R}}}
\def\rmS{{\mathbf{S}}}
\def\rmT{{\mathbf{T}}}
\def\rmU{{\mathbf{U}}}
\def\rmV{{\mathbf{V}}}
\def\rmW{{\mathbf{W}}}
\def\rmX{{\mathbf{X}}}
\def\rmY{{\mathbf{Y}}}
\def\rmZ{{\mathbf{Z}}}

% Elements of random matrices
\def\ermA{{\textnormal{A}}}
\def\ermB{{\textnormal{B}}}
\def\ermC{{\textnormal{C}}}
\def\ermD{{\textnormal{D}}}
\def\ermE{{\textnormal{E}}}
\def\ermF{{\textnormal{F}}}
\def\ermG{{\textnormal{G}}}
\def\ermH{{\textnormal{H}}}
\def\ermI{{\textnormal{I}}}
\def\ermJ{{\textnormal{J}}}
\def\ermK{{\textnormal{K}}}
\def\ermL{{\textnormal{L}}}
\def\ermM{{\textnormal{M}}}
\def\ermN{{\textnormal{N}}}
\def\ermO{{\textnormal{O}}}
\def\ermP{{\textnormal{P}}}
\def\ermQ{{\textnormal{Q}}}
\def\ermR{{\textnormal{R}}}
\def\ermS{{\textnormal{S}}}
\def\ermT{{\textnormal{T}}}
\def\ermU{{\textnormal{U}}}
\def\ermV{{\textnormal{V}}}
\def\ermW{{\textnormal{W}}}
\def\ermX{{\textnormal{X}}}
\def\ermY{{\textnormal{Y}}}
\def\ermZ{{\textnormal{Z}}}

% Vectors
\def\vzero{{\bm{0}}}
\def\vone{{\bm{1}}}
\def\vmu{{\bm{\mu}}}
\def\vtheta{{\bm{\theta}}}
\def\va{{\bm{a}}}
\def\vb{{\bm{b}}}
\def\vc{{\bm{c}}}
\def\vd{{\bm{d}}}
\def\ve{{\bm{e}}}
\def\vf{{\bm{f}}}
\def\vg{{\bm{g}}}
\def\vh{{\bm{h}}}
\def\vi{{\bm{i}}}
\def\vj{{\bm{j}}}
\def\vk{{\bm{k}}}
\def\vl{{\bm{l}}}
\def\vm{{\bm{m}}}
\def\vn{{\bm{n}}}
\def\vo{{\bm{o}}}
\def\vp{{\bm{p}}}
\def\vq{{\bm{q}}}
\def\vr{{\bm{r}}}
\def\vs{{\bm{s}}}
\def\vt{{\bm{t}}}
\def\vu{{\bm{u}}}
\def\vv{{\bm{v}}}
\def\vw{{\bm{w}}}
\def\vx{{\bm{x}}}
\def\vy{{\bm{y}}}
\def\vz{{\bm{z}}}

% Elements of vectors
\def\evalpha{{\alpha}}
\def\evbeta{{\beta}}
\def\evepsilon{{\epsilon}}
\def\evlambda{{\lambda}}
\def\evomega{{\omega}}
\def\evmu{{\mu}}
\def\evpsi{{\psi}}
\def\evsigma{{\sigma}}
\def\evtheta{{\theta}}
\def\eva{{a}}
\def\evb{{b}}
\def\evc{{c}}
\def\evd{{d}}
\def\eve{{e}}
\def\evf{{f}}
\def\evg{{g}}
\def\evh{{h}}
\def\evi{{i}}
\def\evj{{j}}
\def\evk{{k}}
\def\evl{{l}}
\def\evm{{m}}
\def\evn{{n}}
\def\evo{{o}}
\def\evp{{p}}
\def\evq{{q}}
\def\evr{{r}}
\def\evs{{s}}
\def\evt{{t}}
\def\evu{{u}}
\def\evv{{v}}
\def\evw{{w}}
\def\evx{{x}}
\def\evy{{y}}
\def\evz{{z}}

% Matrix
\def\mA{{\bm{A}}}
\def\mB{{\bm{B}}}
\def\mC{{\bm{C}}}
\def\mD{{\bm{D}}}
\def\mE{{\bm{E}}}
\def\mF{{\bm{F}}}
\def\mG{{\bm{G}}}
\def\mH{{\bm{H}}}
\def\mI{{\bm{I}}}
\def\mJ{{\bm{J}}}
\def\mK{{\bm{K}}}
\def\mL{{\bm{L}}}
\def\mM{{\bm{M}}}
\def\mN{{\bm{N}}}
\def\mO{{\bm{O}}}
\def\mP{{\bm{P}}}
\def\mQ{{\bm{Q}}}
\def\mR{{\bm{R}}}
\def\mS{{\bm{S}}}
\def\mT{{\bm{T}}}
\def\mU{{\bm{U}}}
\def\mV{{\bm{V}}}
\def\mW{{\bm{W}}}
\def\mX{{\bm{X}}}
\def\mY{{\bm{Y}}}
\def\mZ{{\bm{Z}}}
\def\mBeta{{\bm{\beta}}}
\def\mPhi{{\bm{\Phi}}}
\def\mLambda{{\bm{\Lambda}}}
\def\mSigma{{\bm{\Sigma}}}

% Tensor
\DeclareMathAlphabet{\mathsfit}{\encodingdefault}{\sfdefault}{m}{sl}
\SetMathAlphabet{\mathsfit}{bold}{\encodingdefault}{\sfdefault}{bx}{n}
\newcommand{\tens}[1]{\bm{\mathsfit{#1}}}
\def\tA{{\tens{A}}}
\def\tB{{\tens{B}}}
\def\tC{{\tens{C}}}
\def\tD{{\tens{D}}}
\def\tE{{\tens{E}}}
\def\tF{{\tens{F}}}
\def\tG{{\tens{G}}}
\def\tH{{\tens{H}}}
\def\tI{{\tens{I}}}
\def\tJ{{\tens{J}}}
\def\tK{{\tens{K}}}
\def\tL{{\tens{L}}}
\def\tM{{\tens{M}}}
\def\tN{{\tens{N}}}
\def\tO{{\tens{O}}}
\def\tP{{\tens{P}}}
\def\tQ{{\tens{Q}}}
\def\tR{{\tens{R}}}
\def\tS{{\tens{S}}}
\def\tT{{\tens{T}}}
\def\tU{{\tens{U}}}
\def\tV{{\tens{V}}}
\def\tW{{\tens{W}}}
\def\tX{{\tens{X}}}
\def\tY{{\tens{Y}}}
\def\tZ{{\tens{Z}}}


% Graph
\def\gA{{\mathcal{A}}}
\def\gB{{\mathcal{B}}}
\def\gC{{\mathcal{C}}}
\def\gD{{\mathcal{D}}}
\def\gE{{\mathcal{E}}}
\def\gF{{\mathcal{F}}}
\def\gG{{\mathcal{G}}}
\def\gH{{\mathcal{H}}}
\def\gI{{\mathcal{I}}}
\def\gJ{{\mathcal{J}}}
\def\gK{{\mathcal{K}}}
\def\gL{{\mathcal{L}}}
\def\gM{{\mathcal{M}}}
\def\gN{{\mathcal{N}}}
\def\gO{{\mathcal{O}}}
\def\gP{{\mathcal{P}}}
\def\gQ{{\mathcal{Q}}}
\def\gR{{\mathcal{R}}}
\def\gS{{\mathcal{S}}}
\def\gT{{\mathcal{T}}}
\def\gU{{\mathcal{U}}}
\def\gV{{\mathcal{V}}}
\def\gW{{\mathcal{W}}}
\def\gX{{\mathcal{X}}}
\def\gY{{\mathcal{Y}}}
\def\gZ{{\mathcal{Z}}}

% Sets
\def\sA{{\mathbb{A}}}
\def\sB{{\mathbb{B}}}
\def\sC{{\mathbb{C}}}
\def\sD{{\mathbb{D}}}
% Don't use a set called E, because this would be the same as our symbol
% for expectation.
\def\sF{{\mathbb{F}}}
\def\sG{{\mathbb{G}}}
\def\sH{{\mathbb{H}}}
\def\sI{{\mathbb{I}}}
\def\sJ{{\mathbb{J}}}
\def\sK{{\mathbb{K}}}
\def\sL{{\mathbb{L}}}
\def\sM{{\mathbb{M}}}
\def\sN{{\mathbb{N}}}
\def\sO{{\mathbb{O}}}
\def\sP{{\mathbb{P}}}
\def\sQ{{\mathbb{Q}}}
\def\sR{{\mathbb{R}}}
\def\sS{{\mathbb{S}}}
\def\sT{{\mathbb{T}}}
\def\sU{{\mathbb{U}}}
\def\sV{{\mathbb{V}}}
\def\sW{{\mathbb{W}}}
\def\sX{{\mathbb{X}}}
\def\sY{{\mathbb{Y}}}
\def\sZ{{\mathbb{Z}}}

% Entries of a matrix
\def\emLambda{{\Lambda}}
\def\emA{{A}}
\def\emB{{B}}
\def\emC{{C}}
\def\emD{{D}}
\def\emE{{E}}
\def\emF{{F}}
\def\emG{{G}}
\def\emH{{H}}
\def\emI{{I}}
\def\emJ{{J}}
\def\emK{{K}}
\def\emL{{L}}
\def\emM{{M}}
\def\emN{{N}}
\def\emO{{O}}
\def\emP{{P}}
\def\emQ{{Q}}
\def\emR{{R}}
\def\emS{{S}}
\def\emT{{T}}
\def\emU{{U}}
\def\emV{{V}}
\def\emW{{W}}
\def\emX{{X}}
\def\emY{{Y}}
\def\emZ{{Z}}
\def\emSigma{{\Sigma}}

% entries of a tensor
% Same font as tensor, without \bm wrapper
\newcommand{\etens}[1]{\mathsfit{#1}}
\def\etLambda{{\etens{\Lambda}}}
\def\etA{{\etens{A}}}
\def\etB{{\etens{B}}}
\def\etC{{\etens{C}}}
\def\etD{{\etens{D}}}
\def\etE{{\etens{E}}}
\def\etF{{\etens{F}}}
\def\etG{{\etens{G}}}
\def\etH{{\etens{H}}}
\def\etI{{\etens{I}}}
\def\etJ{{\etens{J}}}
\def\etK{{\etens{K}}}
\def\etL{{\etens{L}}}
\def\etM{{\etens{M}}}
\def\etN{{\etens{N}}}
\def\etO{{\etens{O}}}
\def\etP{{\etens{P}}}
\def\etQ{{\etens{Q}}}
\def\etR{{\etens{R}}}
\def\etS{{\etens{S}}}
\def\etT{{\etens{T}}}
\def\etU{{\etens{U}}}
\def\etV{{\etens{V}}}
\def\etW{{\etens{W}}}
\def\etX{{\etens{X}}}
\def\etY{{\etens{Y}}}
\def\etZ{{\etens{Z}}}

% The true underlying data generating distribution
\newcommand{\pdata}{p_{\rm{data}}}
% The empirical distribution defined by the training set
\newcommand{\ptrain}{\hat{p}_{\rm{data}}}
\newcommand{\Ptrain}{\hat{P}_{\rm{data}}}
% The model distribution
\newcommand{\pmodel}{p_{\rm{model}}}
\newcommand{\Pmodel}{P_{\rm{model}}}
\newcommand{\ptildemodel}{\tilde{p}_{\rm{model}}}
% Stochastic autoencoder distributions
\newcommand{\pencode}{p_{\rm{encoder}}}
\newcommand{\pdecode}{p_{\rm{decoder}}}
\newcommand{\precons}{p_{\rm{reconstruct}}}

\newcommand{\laplace}{\mathrm{Laplace}} % Laplace distribution

\newcommand{\E}{\mathbb{E}}
\newcommand{\Ls}{\mathcal{L}}
\newcommand{\R}{\mathbb{R}}
\newcommand{\emp}{\tilde{p}}
\newcommand{\lr}{\alpha}
\newcommand{\reg}{\lambda}
\newcommand{\rect}{\mathrm{rectifier}}
\newcommand{\softmax}{\mathrm{softmax}}
\newcommand{\sigmoid}{\sigma}
\newcommand{\softplus}{\zeta}
\newcommand{\KL}{D_{\mathrm{KL}}}
\newcommand{\Var}{\mathrm{Var}}
\newcommand{\standarderror}{\mathrm{SE}}
\newcommand{\Cov}{\mathrm{Cov}}
% Wolfram Mathworld says $L^2$ is for function spaces and $\ell^2$ is for vectors
% But then they seem to use $L^2$ for vectors throughout the site, and so does
% wikipedia.
\newcommand{\normlzero}{L^0}
\newcommand{\normlone}{L^1}
\newcommand{\normltwo}{L^2}
\newcommand{\normlp}{L^p}
\newcommand{\normmax}{L^\infty}

\newcommand{\parents}{Pa} % See usage in notation.tex. Chosen to match Daphne's book.

\DeclareMathOperator*{\argmax}{arg\,max}
\DeclareMathOperator*{\argmin}{arg\,min}

\DeclareMathOperator{\sign}{sign}
\DeclareMathOperator{\Tr}{Tr}
\let\ab\allowbreak

% packages
\usepackage[colorlinks]{hyperref} % for links
%\usepackage[backend=biber]{biblatex} % for references
%\addbibresource{bib.bib} % for references

\usepackage{graphicx} % for embedding graphics
\usepackage{booktabs} % for pretty tables
\usepackage{lipsum} % for gibberish text
\usepackage{mathtools}
\usepackage{wrapfig}
\usepackage{natbib}
\usepackage{floatrow}
\bibliographystyle{abbrvnat}
\setcitestyle{authoryear, citesep={;}, aysep={,},yysep={;}}
\setcitestyle{square}


\doublespacing
\usepackage{xcolor}
\definecolor{amaranth}{HTML}{E83151}
\definecolor{russianviolet}{HTML}{330036}
\definecolor{darkblue}{HTML}{645E9D}
\hypersetup{
    colorlinks=true,
    linkcolor=russianviolet,
    citecolor=darkblue,
    urlcolor=amaranth}
\newcommand{\xhdr}[1]{\noindent{{\bf #1.}}}
\usepackage{listings}
\usepackage{pythonhighlight}
\usepackage{makecell}
\usepackage{tabularx}

\usepackage{cleveref}
\usepackage{amssymb}
\usepackage{amsthm}

\usepackage[ruled,vlined]{algorithm2e}
\newcommand\mycommfont[1]{\footnotesize\ttfamily\textcolor{blue}{#1}}
\SetCommentSty{mycommfont}

\newcommand{\boldQ}{\mathbf{Q}}
\newcommand{\boldP}{\mathbf{P}}
\newcommand{\Qc}{\mathcal{Q}}
\newcommand{\Pc}{\mathcal{P}}
\newcommand{\Rc}{\mathcal{R}}
\newcommand{\Pun}{\boldP^\star}

\newcommand{\method}{Detectron}

\theoremstyle{definition}
\newtheorem{definition}{Definition}
\newtheorem{assumption}{Assumption}
\newtheorem{theorem}{Theorem}
\newtheorem{lemma}{Lemma}

\newtheorem{manualtheoreminner}{Theorem}
\newenvironment{manualtheorem}[1]{%
    \renewcommand\themanualtheoreminner{#1}%
    \manualtheoreminner}
    {\endmanualtheoreminner}

% author data
\author{Tom Ginsberg}
\title{Identifying and Characterizing High Dimensional Covariate Shift with Learning Models}
\degree{Master of Science}
\department[]{Department of Computer Science}
\gradyear{2023}

\makeatletter
\renewcommand{\@makechapterhead}[1]{%
        {\noindent\raggedright\normalfont% Alignment and font reset
    \huge\bfseries \@chapapp\space\thechapter~~#1\par\nobreak}% Formatting
    \vspace{\baselineskip}% ...just a little space
}
\makeatother

% Document
\begin{document}
    \maketitle
    \begin{abstract}
        The ability to quickly and accurately identify covariate shift at test time is a critical and often overlooked component of safe machine learning systems deployed in high-risk domains.
        While methods exist for detecting when predictions should not be made on out-of-distribution test examples, identifying distributional level differences between training and test time can help determine when a model should be removed from the deployment setting and retrained.
        In this thesis, we explore modern and foundational methods for identifying and characterizing distributional shift in high dimensional data, in particular where such data is treated as the covariates to a learning model --- this type of distribution shift is known formally as covariate shift.
        We go on to provide a new definition for \textit{harmful covariate shift} ({\small HCS}) that goes beyond ideas from standard learning theory to give a richer insight on when covariate shift may hurt the performance of classification models.
        Motivated from our definition, we propose a method, the Detectron, to detect {\small HCS} based on the discordance between an ensemble of \textit{constrained disagreement classifiers} (CDCs)  trained to agree on training data and disagree on test data.
        We derive a loss function for training CDCs and show that their disagreement rate and predictive entropy represent powerful discriminative statistics for {\small HCS}.
%        Furthermore, we present tight finite sample shift detection guarantees in an idealized setting.
        Empirically, we demonstrate that the CDC learning algorithm produces model with behaviour that aligns well with our desideratum.
        Finally we showcase the ability of the Detectron to detect {\small HCS} with statistical certainty on a variety of high-dimensional datasets.
        Across numerous domains and modalities, we show state-of-the-art performance compared to existing methods, particularly when the number of observed test samples is small.
    \end{abstract}

    \begin{acknowledgements}
        I would first and foremost like to thank my graduate supervisor Professor Rahul G. Krishnan,
        for his continuous support and mentorship throughout my studies, as well as Professor Murat Erdogdu for
        his role as an additional reader for this thesis.
        I would also like to thank my original graduate cohort, Vahid Balazadeh, and Michael Cooper, for their
        help brainstorming ideas, useful feedback, research advice, squash playing and all
        around creation of an amazing lab environment.
        Additional thanks to others that provided helpful advice and feedback include:
        Edward De Brouwer, Aslesha Pokhrel, Zhongyuan Liang, Adnan Mohd, Ian Shi, Asic Chen and Stephan Rabanser.
    \end{acknowledgements}
    \tableofcontents
    \listoftables
    \listoffigures

    \chapter[Introduction]{Introduction: Covariate Shift}
    \label{ch:intro}

    Machine learning models operate on the assumption, albeit incorrectly that they will be deployed on data distributed identically to what they were trained on.
The violation of this assumption is known as distribution shift and can often result in significant degradation of performance~\citep{cifar101,cifar10C,learnundercov, failloud, mindperfgap, trustuncert}.
There are several cases where a mismatch between training and deployment data results in very real consequences on human beings.
In healthcare, machine learning models have been deployed for predicting the likelihood of sepsis.
Yet, as~\citep{10.1001/jamainternmed.2021.3333} show, such models can be miscalibrated for large groups of individuals, directly affecting the quality of care they experience.
The deployment of classifiers in the criminal justice system~\citep{AIissend45:online}, hiring and recruitment pipelines~\citep{Amazonsc17:online} and self-driving cars~\citep{selfdrive} have all seen humans affected by the failures of learning models.
The need for methods that quickly detect, characterize and respond to distribution shift is, therefore, a fundamental problem in trustworthy machine learning.
For practitioners, regulatory agencies and individuals to have faith in deployed predictive models without the need for laborious manual audits, we need methods for the identification of distribution shift
that are \textit{sample-efficient} (identifying shifts from a small number of samples), \textit{informed} (identifying shifts relevant to the domain and learning algorithm), \textit{model-agnostic}
(identifying shifts regardless of the functional class of the predictive model) and \textit{statistically sound} (identifying true shifts while avoiding false positives with high-confidence).

In our work, we study a special case of distribution shift, commonly known as \textit{covariate shift}, which considers shifts only in the distribution of input data $\Pc_X \neq \Qc_X$ while the relation between the inputs and outputs remains fixed $\Pc_{Y|X} = \Qc_{Y|X}$.
In a standard deployment setting where ground truth labels are not available, estimating any properties of the true distribution $\Qc_{Y|X}$ is fundamentally impossible without strong assumptions.
As a consequence covariate shift is the only type of distribution shift that can be identified.

We build off recent progress in understanding model performance under covariate shift using the PQ-learning framework~\citep{pqlearn}, a framework for selective classifiers that may either predict on or reject a given sample, that provides strong performance guarantees on arbitrary test distributions.
Our work uses and extends PQ-learning to develop a practical, model-based hypothesis test, named \textit{the \method}, to identify potentially harmful covariate shifts given any existing classification model already in deployment.

Our work makes the following key contributions:

\begin{itemize}
    \item We show how to construct an ensemble of classifiers that maximize out-of-domain disagreement while behaving consistently in the training domain.
    We propose the \emph{disagreement cross entropy} for models learned via continuous gradient-based methods (e.g., neural networks), as well as a generalization for those learned via discrete optimization (e.g., random forest).
    \item We show that the rejection rate and the entropy of the learning ensemble can be used to define a model-aware hypothesis test for covariate shift, the \method,\ that in idealized settings can provably detect covariate shift.
    \item On high-dimensional image and tabular data, using both neural networks and gradient boosted decision trees, our method outperforms state-of-the-art techniques for detecting covariate shift, particularly when given access to as few as ten test examples.
\end{itemize}

\begin{figure}[!htb]
    \centering
    \includegraphics[width=\linewidth]{images/detectron.pdf}
    \caption{\small \textbf{Detectron (PQ Learning for Covariate Shift Detection):} Starting with a base classifier trained on labeled samples from distribution $\Pc$ we
    train new \textit{\textbf{C}onstrained \textbf{D}isagreement \textbf{C}lassifiers} (CDCs) on a small set of observed unlabeled samples from a new distribution $\Qc$.
    CDCs aim to maximize classification disagreement on $\Qc$ while constrained to agree with the base classifier on $\Pc$. The rate $\phi$ that CDCs disagree,
        as well as their entropy, on test data is a powerful and sample efficient statistic for identifying covariate shift $\Pc\neq \Qc$.}
    \label{fig:detectron}
\end{figure}


    \chapter[Background: Covariate Shift]{Background: Covariate Shift}\label{ch:related}

    Covariate shift is the tendency for a covariate distribution at test time $p_{\text{test}}(x)$ to differ from that seen
during training $p_{\text{train}}(x)$ while the underlying prediction concept $y$ remains fixed~\citep{covbook}
\[p_{\text{train}}(x)\neq p_{\text{test}}(x) \text{ while } p_{\text{train}}(y|x)=p_{\text{test}}(y|x)\]
The problem of detecting variate shift is important at a fundamental level in machine learning.
From a classical learning theory perspective, predictive ML algorithms that operate on unseen data
can only claim reasonable performance guarantees when such data is exchangeable with the labeled samples which they were trained and tested on~\citep{Haussler90probablyapproximately}.
The violation of this assumption in general is known as distribution shift.
However, in a typical deployment setting where only covariates are observed one cannot detect changes in $p(y)$ or $p(y|x)$ leaving covariate shift as the only remaining phenomenon identifiable from data.
The related problems of \textit{label shift} ({shift in $p(y)$}) and \textit{concept shift} (shift in $p(y|x)$) pose an additional problems for maintaining the robustness of deployed ML systems, but are not the topic of this work.
To understand contemporary methods for detecting covariate shift in the high dimensional data that is typical of the covariates found in real-world ML systems we must first take a segway back to the underlying concepts in probability and statistics.


\section{Two Sample Statistical Testing}
\textit{If the reader is familiar with the subject of two sample testing they may continue to \autoref{sec:cov-high-dimensions}.}
\smallbreak\noindent
The problem of detecting when two distributions $\Pc$ and $\Qc$ over $\mathcal{X}$ are different given only finite samples $\boldP=\{X_1,\ldots X_n\}\overset{\text{iid}}{\sim} {\Pc}$ and $\boldQ=\{\tilde{X}_1,\ldots \tilde{X}_m\}\overset{\text{iid}}{\sim} {\Qc}$ is one of the most fundamental problems in statistics.
The general problems seeks to rule out a \textit{null hypothesis} $\mathcal{H}_0:\ \Pc=\Qc$ in favor of an alternative $\mathcal{H_a}:\ \Pc\neq \Qc$ at a given significance level $\alpha$.
The significance level is equivalent to the false positive rate or probability of making a type I error, which fixes a threshold for the test such that when $\Pc=\Qc$, $\mathcal{H}_0$ is ruled out with probability no more than $\alpha$.
The power of a statistical test, often denoted as $1-\beta$ where $\beta$ is the probability of making a type II error, is the probability that the test correctly rules out $\mathcal{H}_0$ while still maintaining a significance level of $\alpha$.
While less standard in statistics literature, it is natural to measure the area under the receiver operating characteristics (AUROC) as area under curve $1-\beta$ as a function of $\alpha$ for $\alpha\in [0,1]$.
For a more rigours mathematical underpinning of two sample testing see section 2.1 in ~\citeauthor{twosampletestingbackground}.
In the following subsections we highlight several important approaches to two sample testing that will be used in much of the later methodologies.

\smallbreak
\subsection*{Kolmogorov–Smirnov test} An elegant and robust two sample test that requires no assumptions on the distributions $\Pc$ and $\Qc$ besides $\mathcal{X}=\mathbb{R}$ is the {Kolmogorov–Smirnov test} (KS test)~\citep{kolmogorov1933sulla}.
The KS test leverages the result of the Glivenko–Cantelli theorem~\citep{glivenko1933sulla} to construct a test statistic defined by the largest absolute difference between the two empirical cumulative distribution functions across all observed values.
As the Glivenko–Cantelli theorem is simple yet fundamental to understanding the KS test we present it in full below:
\begin{theorem}
    [Glivenko–Cantelli~\citep{glivenko1933sulla}]
    Assume that $X_1,X_2,\dots$ are independent and identically-distributed random variables in $\mathbb {R}$  with common cumulative distribution function $F(x)$.
    The empirical distribution function for $X_{1},\dots ,X_{n}$ is defined by
    \begin{equation}
        F_n(x)=\frac{1}{n} \sum_{i=1}^n I_{\left[X_i, \infty\right)}(x)=\frac{1}{n}\left|\left\{1 \leq i \leq n \mid X_i \leq x\right\}\right|\label{eq:empdist}
    \end{equation}
    where $I_{C}$ is the indicator function of the set $C$.
    \noindent
    For every (fixed) $x$, $F_n(x)$ is a sequence of random variables which converge to $F(x)$ almost surely by the strong law of large numbers.
    \begin{equation}
        \left\|F_n-F\right\|_{\infty}=\sup _{x \in \mathbb{R}}\left|F_n(x)-F(x)\right| \longrightarrow 0 \text { almost surely }
    \end{equation}
\end{theorem}

To perform a KS two sample test, one computes the exact probability under the null hypothesis $\Pc=\Qc$ of obtaining a rarer value compared to the observation of the test statistic
\[D_{\boldP,\boldQ}\triangleq \sup_{x \in \mathbb{R}}\left|F_{\boldP}(x)-F_{\boldQ}(x)\right|\]
Where $F_{\mathbf{S}}$ is the empirical CDF for a set of samples $\mathbf{S}$.

Computing the probability of observing a rarer test statistic is non-trivial and in general maps to the problem of counting the number of distinct paths that pass outside the specified diagonal on a $|\boldP| \times |\boldQ|$ rectangular grid~\citep{drew2000computing}.
However, for large $\boldP$ and $\boldQ$ asymptotic approximations exist that can greatly reduce the computational complexity.
A shortcoming of the KS test is that it sacrifices statistical power in order to be more flexible to handle arbitrary differences between distributions.

\subsection*{Binomial test} While the KS test present a general approach to univariate two sample testing, we can often use partial knowledge over the distribution that generates our samples to design of a more powerful test;
the binomial test is one such example.
The test considers a binomially distributed random variable $X$ with unknown rate $q$ i.e., $X\sim \text{Binomial}(n, q)$ for which we observe a single sample $x$ (hence $\mathcal{X}=\{0,\ldots, n\}$).
Since the binomial distribution is defined as a sum of i.i.d.\ Bernoulli random variables with the same rate, $x$ may equivalently be interpreted as a set of $n$ samples of which $x$ are $1$ and $n-x$ are $0$.
We wish to either confirm or rule out that $x$ was sampled from a baseline binomial distribution with known rate $p$.
We do this by computing the probability of observing an event at least as rare as $X=x$ under the null hypothesis that $p=q$.
This quantity can be computed exactly using the symmetry of the binomial distribution.
\begin{align*}
    \mathbb{P}_{X\sim \text{Bin}(n, p)}(X \text{ as least as rare as } x) &= 2\times \mathbb{P}_{X\sim \text{Bin}(n, p)}(X \geq x)\\
    &= 2 \sum_{k=x}^{n} \mathbb{P}[X=k] \\
    &= 2 \sum _{k=x}^n (1-p)^{n-k} p^k \binom{n}{k} = 2\frac{B_p(x,n-x+1)}{B(x,n-x+1)}
\end{align*}
Where $B_z(\alpha, \beta)$ is the incomplete Beta function and $B(\alpha, \beta)$ is the beta function.
By the convenient construction of the test one can query the likelihood specifically for one-sided alternatives;
something that is more practical when there is specific knowledge related to the alternative hypothesis.
For example, one can test if a coin is unfairly biased towards heads given an observation of $60/100$ heads, in this case the one-sided binomial test admits a $p$-value of $\approx 0.028$ suggesting that we can rule out at the $5\%$ level that the coin is fair
in favor of it being biased towards heads.
\smallbreak
Do to the simple structure of the binomial test one can derive a simple summation formula for computing the
exact statistical power as a function of $n$, $p$, $q$ and the significance level $\alpha$.
This quantity corresponds to the probability of observing a $p$-value of less than or equal to $\alpha$ when running a
binomial test on sample from a random variable $X\sim \text{Bin}(n,q)$ against a baseline rate $p$.
\begin{align*}
    \mathbb{P}_{Y\sim \text{Bin}(n, q)} \left( \mathbb{P}_{X\sim \text{Bin}(n, p)}(X \geq Y) \leq \frac{\alpha}{2} \right)&= \mathbb{P}_{Y\sim \text{Bin}(n, q)}\left(  \sum_{k=k}^n \binom{n}{k} (1-p)^{n-k} p^k  \leq \frac{\alpha}{2} \right)\\
    &= \mathbb{E}_{Y\sim \text{Bin}(n, q)}\left(  \mathbb{I}\left[\sum_{k=k}^n \binom{n}{k} (1-p)^{n-k} p^k  \leq \frac{\alpha}{2}\right] \right)\\
    &= \sum_{y=0}^n \binom{n}{y} \mathbb{I}\left[  \sum_{k=y}^n \binom{n}{k} (1-p)^{n-k} p^k \leq \frac{\alpha}{2} \right] q^y (1-q)^{n-y}
\end{align*}
We use $\mathbb{I}[\cdot]$ as a binary indicator.
We verify this formula using a simple monte carlo simulation with $p=.5$, $q=.6$, $n=100$ and $\alpha=0.05$.
The statistical power and standard error estimated over $10,000$ draws gives $0.461\pm 0.005$ whereas the above summation yields a value of $0.4621$.
One important note is that under this simple construction the binomial test is not a two sample test, but merely a single sample test.
However, in a setting where one has access to many more sample in $\boldP$ compared to that in $\boldQ$
(which will be the primary setting of the upcoming methodology), one can reasonably estimate the baseline success rate $p$ from $\boldP$ and use the binomial test as is.
In other settings where the assumption of $|\boldP|\gg|\boldQ|$ does not hold, one may resort to Barnard's exact test~\citep{Barnardp88:online} or the often more powerful alternative Fisher's exact test~\citep{fisher_r_a_1922_1449484}.

\subsection*{Integral Probability Metrics \& Maximum Mean Discrepancy}
A general method to define a measure between probability distributions is via an integral probability metric or IPM~\citep{IPMS}.
An IPM is defined as the supremum over a set of functions $F$ on the absolute difference between the expectations under $\Pc$ and $\Qc$.
\begin{equation*}
    \text{IPM}_F(P \| Q)=\sup_{f\in F}|\mathbb{E}_{x\sim P}\left[ f(x)\right] - \mathbb{E}_{x\sim Q}\left[ f(x)\right]|
\end{equation*}
When $F$ is chosen as the set of all functions with bounded $\infty-$norm $F = \left\{f:\|f\|_{\infty} \leq 1\right\}$
the associated IPM results in the well known total variation distance~\citep{ipm-divergences}.

The Maximum Mean Discrepancy (MMD) is formalized as an IPM given $F$ to be the n Reproducing Kernel Hilbert Space $\mathcal{H}_k$ with a kernel $k: \mathcal{X}\times \mathcal{X} \to \mathbb{R}$~\citep{twosampletestingbackground}.
\begin{equation*}
    \operatorname{MMD}\left(\Pc, \Qc ; \mathcal{H}_k\right) = \sup _{f \in \mathcal{H}_k:\|f\|_{\mathcal{H}_k} \leq 1}\left|\mathbb{E}_{X \sim \Pc}[f(X)]-\mathbb{E}_{Y \sim \Qc}[f(Y)]\right|
\end{equation*}
When we chose a characteristic kernel such that $\operatorname{MMD}\left(\Pc, \Qc ; \mathcal{H}_k\right)=0\iff \Pc=\Qc$ \citeauthor{gretton2012kernel} construct a quadratic time MMD estimator using samples $\boldP$ and $\boldQ$ of sizes $n$ and $m$ respectively:
\begin{equation}
    \begin{aligned}
        \widehat{\mathrm{MMD}}^2\left(\boldP, \boldQ ; \mathcal{H}_k\right) = \frac{1}{n(n-1)} \sum_{1 \leq i \neq i^{\prime} \leq n} k\left(\boldP_i, \boldP_{i^{\prime}}\right) &+\frac{1}{m(m-1)} \sum_{1 \leq j \neq j^{\prime} \leq m} k\left(\boldQ_j, \boldQ_{j^{\prime}}\right) \\
        &-\frac{2}{m n} \sum_{i=1}^n \sum_{j=1}^m k\left(\boldP_i, \boldQ_j\right)
    \end{aligned}
\end{equation}
Existing work for detecting high dimensional distribution shift using MMD estimators focus on the design of kernels.
\citeauthor{failloud} considers the squared exponential kernel $k(x, y)=\exp{\left(\sigma^{-1} \|x-y\|^2\right)}$, while~\citeauthor{liu2020learning} explore of use of deep neural networks for building more statistically powerful kernels.

While MMD presents an estimator for measuring the difference in distributions, it does not directly lead to a rigorous two sample test with a bounded significance level, to achieve this we will need to turn to the methodology of permutation testing presented in the next subsection.

\subsection*{Permutation Testing}
Permutation testing~\citep{fisher_1991} is a method to guarantee the significance level of any test by first estimating a quantile of the distribution of a test statistic when the null hypothesis $\mathcal{H}_0$ is true ($\Pc=\Qc$).
For some tests like KS and Binomial this quantile can be computed analytically or using an exact algorithm, but in the general case it cannot, motivating the need for alternative methods.

Suppose we have a test statistic function $T: \mathcal{X}^n\times \mathcal{X}^m\to \mathbb{R}$.
In the context of two sample testing $T$ maps a set $\boldP$ and $\boldQ$ to a real value which we assume to be some heuristic measure
of the distributional difference between $\Pc$ and $\Qc$ based on the finite sample observations.
To construct a one-sided two sample test with $T$ we must first determine a significance threshold $\tau$ such that when $\Pc=\Qc$ we have a probability of at least $1-\alpha$
that $T(\boldP, \boldQ) < \tau$ (or $T(\boldP, \boldQ) > \tau$ depending on the direction of the test).
To estimate $\tau$ we take advantage of the exchangeability of samples under $\mathcal{H}_0$.
If indeed $\Pc=\Qc$ then it should not matter which group of the total $n+m$ samples we call $\boldP$ and which we call $\boldQ$, so the strategy is to
combine, permute and redivide $\boldP$ and $\boldQ$ and feed them into $T$ many times, computing $\tau$ as the $1-\alpha$ quantile of all observations of the test statistic.
If the test performed on the original partitions exceeds $\tau$ (e.g., $T(\boldP, \boldQ) \geq \tau$) we can rule out $\mathcal{H}_0$ with an exact significance level $\alpha$.
We note that as we are estimating $\tau$ from finite data, it itself incurs statistical estimation error which in turn adjusts the true significance level of a test~\citep{NEURIPS2019_8fb21ee7}.
This is a minor detail which is ignored by several modern approaches for two sample testing that employ permutation tests~\citep{liu2020learning, zhao2022comparing}.


\section{Approaches for High Dimensional Data}\label{sec:cov-high-dimensions}
\subsection*{Dimensionality Reduction Approaches}
Many methods for detecting shift in high-dimensional data simply apply dimensionality reduction (DR) techniques followed by standard two sample tests~\citep{failloud}.
\citeauthor{failloud} perform an evaluation on several DR techniques including, PCA, random feature projections, trained and untrained auto-encoders and the output features of a classification model.
On each DR method they test the performance of univariate KS-tests on each dimension of the reduced data distribution aggregated using Bonfferoni correction as well as MMD using the squared exponential kernel.
\citeauthor{failloud}'s main takeaway is that using the softmax outputs of a pretrained classifier as low dimensional representations for performing KS-tests, a method known as black box
shift detection (BBSD)~\citep{bbsd}, is effective at confidently identifying several synthetic covariate shifts in imaging data (e.g., crops, rotations) given approximately 200 i.i.d samples.
However, applying statistical tests to non-invertible representations of data can never guarantee to capture arbitrary covariate shifts, as there may always exist multiple distributions that collapse to
the same test statistic~\citep{failuresofgen}.
For instance if considering PCA as the DR of choice an adversary could construct a new data distribution $\Qc$ where samples are drawn by transforming samples from $\Pc$ randomly within the null space of the orthogonal projection.

\subsection*{Learning Theoretic Approaches}
\cite{domainrep} introduces the earliest theoretic framework for identifying and bounding the effect of covariate shift based on discriminative learning with finite samples.
In the foundation work ``Detecting Change in Data Streams'', \citeauthor{atheory} introduce the concept of $\mathcal{A}$-distance as a generalization of the total variation to an arbitrary collection of measurable events $\mathcal{A}$.
\begin{equation}
    d_{\mathcal{A}}\left(\mathcal{P}, \mathcal{Q} \right)\triangleq2 \sup _{A \in \mathcal{A}}\left|\operatorname{Pr}_{\mathcal{D}}[A]-\operatorname{Pr}_{\mathcal{D}^{\prime}}[A]\right|\label{eq:total-var}
\end{equation}
The authors show how various choices of $\mathcal{A}$ can induce certain desired properties, for example when $\Pc$ and $\Qc$ are distribution over the real line and $\mathcal{A}$ is the collection of all intervals $(-\infty, \cdot)$
the $\mathcal{A}$-distance becomes equivalent to the Kolmogorov-Smirnov statistic.
\citeauthor{atheory} go on to prove tight bounds on various instantiations of the $\mathcal{A}$-distance for detecting shifts in an online setting.

\citeauthor{domainrep} develop a learning theoretic extension of \citeauthor{atheory} to show that when they chose a class of events whose characteristic
functions are functions from a set of binary classifiers $F$, the $\mathcal{A}$ distance in connection with VC theory~\citep{vapnik95}
allows for finite sample generalization bounds on the performance of arbitrary decision models from $F$ under covariate shift.
\citeauthor{domainrep} go on to show that the $\mathcal{A}$ distance defined for a binary function class $F$ is equal to
\begin{equation}
    d_F(\Pc, \Qc) \triangleq 2\left(1-2 \min_{f\in F}\text{err}(f)\right)\label{eq:a-dist}
\end{equation}
where $\min_{f\in F}\text{err}(f)$ is the minimum error that a domain classifier from $F$ can achieve on the task of distinguishing samples from $\Pc$ and $\Qc$
(i.e., if $\Pc=\Qc$ the best domain classifier will have error of 0.5 and $d_F(\Pc, \Qc)=0$ and if $\Pc$ and $\Qc$ can be perfectly discriminated by some $f\in F$ the $d_F(\Pc, \Qc)$ is maximized and equal to $2$).
This conclusion motivates the classifier two sample test (CTST)~\citep{paz2017revisiting} to detect shift by first training a domain classifier $d: \mathcal{X}\to \{0,1\}$ on a set of samples from $\Pc$ and $\Qc$ then
running a binomial test to reject the null hypothesis that $d$ achieves an accuracy of 0.5 on unseen data.
\citeauthor{domainrep} inspired many ideas for unsupervised domain adaption including the popular method \textit{domain adversarial domain adaptation (DANN)}~\citep{ganin2016domain}
which aims to learn robust representations of data for classification by explicitly regularizing for a small $d_F$ on the representation space.

\subsubsection*{Learning Model Based Approaches}
More recent approaches for covariate shift detection including, deep kernel MMD~\citep{liu2020learning} and H-Divergence~\citep{zhao2022comparing} focus on training learning models with objectives optimized specifically for statistical testing in a way that goes beyond CTSTs.

\citeauthor{liu2020learning} show how to train a deep feature extractor $\phi: \mathcal{X}\to\mathbb{R}^d$ to
optimize the power of an MMD test using a simple kernel (e.g., squared exponential) whose inputs are feature vectors tom $\phi$.
\citeauthor{zhao2022comparing} introduce a new family of distributional divergences (H-Divergences) that builds off the concept of H-entropy~\citep{10.1214/aoms/1177704567}
\[H_\ell \triangleq \inf_{a\in \mathcal{A}}\mathbb{E}_{X\sim\Pc}[\ell(X, a)]\]
The H-entropy is defined on a distribution $\Pc$ as the optimal expected action with respect to loss function $\ell$ and set of possible actions $\mathcal{A}$.
For example if $\ell$ if the reconstruction loss of an autoencoder $f_\theta$ and $\mathcal{A}$ is the set of all possible parameter configurations $\theta$, $H_\ell$ corresponds to the expected reconstruction
loss of $f_{\theta^\star}$ where $\theta^\star$ is the optimal choice of parameters.

\citeauthor{zhao2022comparing} present a general definition of the H-Divergence using a function whose arguments are restricted to be $H_\ell\left( \frac{p+q}{2} \right) - H_\ell(p)$ and $H_\ell\left( \frac{p+q}{2} \right) - H_\ell(q)$.


In our work we take a transductive learning approach and construct a method to directly use the structure of a supervised classification problem to improve the statistical power for detecting shifts.


\section{Out of Distribution Detection}\label{sec:out-of-distribution-detection}
Out of distribution (OOD) detection focuses on identifying when a specific data point $x'$ admits low likelihood under the original training distributions
$(p_{\text{train}}(x') \approx 0)$---a useful tool to have at inference time.~\cite{densityratio, densityofstates} represent a broad class of work that uses density estimation
to pose the identification of covariate shift as anomaly detection.
However, in finite samples, density estimation for high-dimensional data can be difficult which in turn affects the accuracy of anomaly detection~\citep{failuresofgen}.
Still the discipline of OOD detection has seen several recent successes, including ODIN~\citep{ODIN}, Deep Mahalanobis Detectors~\citep{mahalano}
and, Gram Matrices~\citep{grammat} which all directly use the predictive model (e.g., information from the intermediate representations of neural networks) to create a real valued score function $\phi: X\to \mathbb{R}$
which attempts to map data points to a real number near zero if the datapoint is in the training distribution and far from zero if the datapoint is out of distribution.
Such methods are largely based on heuristics on the manifold of neural networks offering little to no theoretical guarantees on detecting subtle types of covariate shifts encountered in real-world settings.
Furthermore, the majority of methods in this space have been designed exclusively for deep neural networks, an uncommon modelling choice particularly for tabular data~\citep{tabdatasurvey}.

Despite these shortcomings, OOD methods can be readily applied for the problem of covariate shift detection under the same principle that governs Integral Probability Metrics (IPMs)~\citep{IPMS}; namely that if two
distributions are identical, any function should have the same expectation under both distributions.
Hence testing for a difference in expected OOD scores between a source and target dataset is a valid hypothesis test for covariate.
However, as OOD scores are designed to only look at a single sample level, instead of treating a set of samples as a whole one would not expect such a test to yield high statistical power for near distribution shifts.


\section{Uncertainty Estimation}\label{sec:uncertainty-estimation}
Related to OOD, uncertainty estimation concerns developing models that identify sources of uncertainty in their predictions~\citep{ensemble, trustuncert}.
A non exhaustive list of approaches to model uncertainty using predictive models are Gaussian Processes~\citep{williams1995gaussian}, measuring the
predictive variance or entropy of Deep Ensembles~\citep{ensemble} or Monte Carlo Dropout~\citep{dropout}, Bayesian Neural Networks~\citep{baysneural}, and Evidential Learning~\citep{evidential, evclass}.
Naturally, uncertainty should be large when samples are OOD implying that one could consider testing for covariate shift using the same principle as identified for OOD scores.
However, ~\citep{trustuncert} perform a large-scale empirical comparison of uncertainty estimation methods and find that while deep ensembles generally provide the best results, the quality of uncertainty estimations, regardless of method, consistently degrades with increasing covariate shift.


\section{Selective Classification and PQ Learning}\label{sec:selective-classification-and-pq-learning}
Selective classification concerns building classifiers that may either predict on or reject on test samples~\citep{selectivenet}.
% One can turn a method for OOD detection or uncertainty estimation into a selective classifier.
% However,~\citep{failuresofgen} show that this is often not guaranteed to be valid for arbitrary distribution shifts.
Recent work by~\citep{pqlearn} develops a formal framework known as PQ learning which extends probably approximately correct (PAC) learning~\citep{Haussler90probablyapproximately} to arbitrary test distributions by allowing for selective classification.
While PAC learning concerns the development of a classifier with a bounded finite-sample error rate on its training distribution, PQ learning seeks a selective classifier with jointly bounded finite-sample error and rejection rates on arbitrary test distributions.
The Rejectron algorithms proposed therein builds an ensemble of models that produce different outputs relative to a perfect baseline on a set of unlabeled test samples.
% While the original algorithm was designed for binary classification in the noiseless setting, follow-up work~\citep{slicendice} has relaxed the latter assumption to allow for one-sided noise.
We provide a summary of the original Rejectron algorithm in the supplementary material (see \autoref{sec:rejectron}).
PQ-learning represents a major theoretical leap for learning guarantees under covariate shift;
however, the majority of the underlying ideas have not been implemented/tested experimentally using real-world data.
We show how to build a PQ learner by generalizing the Rejectron algorithm, overcoming several limitations and assumptions made by the original work including extending beyond simple binary classification to general multiclass/multilabel tasks and reducing the number of samples required for learning at each iteration.
We go on to show how a PQ learner can be used to characterize covariate shifts in real-world data.


    \chapter[Methodology: The Detectron]{Methodology: The Detectron}\label{ch:detectron}

    \section{Problem Setup}\label{sec:problem-setup}
Let $f: X\to \mathbb{R}^N $ be a classifier from a function class $F$ that maps from space of covariates $X$ to a discrete probability distribution over classes $Y=\{1,\ldots,N\}$ i.e.,
\[f(x)=[f(x)_1, f(x)_2, \ldots, f(x)_N]^\top\text{ s.t. }f(x)_i \geq 0\ \forall\ i\in Y \text{ and } \|f(x)\|_1 = 1\]
We denote the prediction of a model $f(x)$ as $\hat{f}(x)\coloneq \argmax_{i\in \{1,\dots,N\} }f(x)$ which corresponds to the class with the maximum predicted probability.
We assume $f$ was trained on a dataset of \textbf{labeled} samples $\boldP = \{(x_i,y_i)\}_{i=1}^n$ where each $x_i$ is drawn identically from a distribution $\Pc$ over $X$, and
each label $y_i \in Y$ corresponds to the ground truth label for the classification problem of interest.
In deployment, $f$ is made to predict on new \textbf{unlabeled} samples $\boldQ = \{\tilde{x}_i\}_{i=1}^m$ ($\tilde{x}_i$ from a distribution $\Qc$ over $X$.
Our high level goal is to determine whether $f$ may be trusted to do so accurately.
More specifically, the problem we address is how to automatically detect, with high statistical power, from only a \textit{small} set of samples $\boldQ$,
if the new covariate distribution $\Qc$ has shifted from $\Pc$.
In addition, we wish to detect true shifts efficiently and with a provably correct lower bound on false positive rate (the rate at which we detect shift when in reality there is no shift).
%The false positive rate is also referred to as the probability of making a type II error, or the significance level and is denoted by the symbol $\alpha$.

Our problem will differ from a standard two sample testng scenario in several critical ways (1) we may assume access to many more samples from $\Pc$
compared to $\Qc$; motivated by the setting where we wish to detect shift quickly, and hence from as few samples as possible
(2) we have access to a robust classifier $f$ that performs \textit{reasonably well} (e.g. could realistically be deployment) on held out samples from $\Pc$ as well as the learning algorithm $L$ used to train it.
(3) We seek to only detect shifts that could reduce classification robustness of $f$ --- we refer to this shift characterization \textit{harmful covariate shift} and elaborate on it in the next section.

\clearpage
\section{Harmful Covariate Shift}\label{sec:harmful-covariate-shift} A shift in the data distribution is not always harmful.
In many practical problems, a practitioner may use domain knowledge to embed invariances with the explicit goal of ensuring the predictive performance of a classifier does not, by construction, change under certain shifts.
This may be done directly via translation invariance in convolutional neural networks, permutation invariance in DeepSets~\citep{deepsets} or indirectly via data augmentation or domain adaptation.
Other model invariances are often learned naturally due to the structure of a given dataset and associated decision problem.

\begin{wrapfigure}{r}{0.5\textwidth}
    \vspace{-8mm}
    \begin{center}
        \includegraphics[width=\linewidth]{images/mnist_corrupted.pdf}
    \end{center}
    \vspace{4mm}
    \caption{\small Blacking out the corners in MNIST is an example of a \textit{non harmful} covariate shift with respect to a simple CNN that achieves near 100\% accuracy on both sets.
    A likely explanation for why this shift is not harmful is because MNIST images contain content primarily in the center, leading to classifiers implicitly learning an invariance to the corners
    of the image.}
    \label{fig:mnist}
    \vspace{-8mm}
\end{wrapfigure}
For example, we train a simple CNN using the LeNet architecture on the MNIST dataset as $\boldP$ and test it on a modified version where a $5\times 5$ pixel region at each corner of image is blacked out $\boldQ$ (see \autoref{fig:mnist}).
We find that when tested on a set of $10,000$ unseen examples the accuracy ($\pm$ standard error) on the original images is $98.48\pm 0.12 \%$ and $98.09\pm 0.14 \%$ on those with blacked out corners (i.e., nearly identical).
A plausible explanation for this observation is that the content useful for explaining MNIST is found near the center of the image,
which encourages a robust classifier to learn invariance to the content of the image near the corners.
However, in terms of pure statistics one could not argue that there is no covariate shift between $\Pc$ and $\Qc$.
To support this claim we run the simplest possible two sample test where we take the mean intensity of each image in $\boldP$ and $\boldQ$ using only a sample size of $|\boldP|=|\boldQ|=10$ and
run a Kolmogorov Smirnov Test~\citep{kstest} over $1,000$ permutations to calculate a test power of $0.981\pm 0.004$ at the strongest significance level $\alpha=0$.

The MNIST example motivates the point that practical heuristics, and problem structure can lead to models generalizing to a more broad range of distributions than can be characterized by just the training set.
And furthermore, that when two sample tests are decoupled with respect to the heuristics and structures of the domain, they lose their informative abilities.
To formalize this notion we define the informal idea of the \textit{generalization set} $\Rc$ as the set of distributions, beyond simply the training distribution, that a model generalizes to.
$\Rc$ is an abstract object which may in general depend on the model architecture/function class $F$, learning algorithm $L$ and training dataset $\boldP$.
While $\Rc$ is difficult if not impossible to characterize exactly, we seek a practical method for detecting shift that is explicitly tied to $\Rc$.

Our approach is based both on PQ learning, from learning theory and learning dynamics.
A one sentence summary is: if we can find a set of classifiers $\{g_1,\ldots, g_\aleph\}\subseteq F$ that achieve the same generalization
set as $f$ but behave inconsistently with respect to $f$ on samples from a distribution $\Qc$, then $\Qc$ must not be a member $\Rc$.
Since it is not possible to know $\Rc$ for a given classifier we introduce a more practical definition of harmful covariate shift
based on the learning algorithm and source dataset --- whose complex interaction is what induces $\Rc$.

\begin{definition}
    [ Harmful Covariate Shift (HCS) ]
    Let $F$ be a family of decision functions that are learnable via a learning algorithm ${L}$ from a set of samples drawn from a source distribution $\Pc$.
    We say a covariate shift from distributions $\Pc \to \Qc$ over $X$ is $(\ell, \alpha, L, \Pc)$-harmful,
    if there exists a finite subset of models of two or more models $\mathbf{f} \subseteq F$ that achieve a source domain loss $\mathbb{E}_{\Pc}[\ell(f(x), y)] \leq \alpha$ for all $f\in \mathbf{f}$ while
    being more likely to disagree with each other on an unseen sample from $\Qc$ compared to $\Pc$.
    \begin{equation}
        \begin{aligned}
            \exists\ \mathbf{f}\subseteq F, &\text{ s.t. } \forall f\in \mathbf{f}\ \ \mathbb{E}_{\Pc}[\ell(f(x), y)] \leq \alpha \text{ and } \\
            &\mathbb{P}_{x\sim \Qc}(\exists\ f_i, f_j \in \mathbf{f}\text{ s.t. }\hat{f}_i(x)\neq \hat{f}_j(x)) > \mathbb{P}_{x\sim \Pc}(\exists\ f_i, f_j \in \mathbf{f}\text{ s.t. }\hat{f}_i(x)\neq \hat{f}_j(x))
        \end{aligned}\label{eq:hcs}
    \end{equation}
    In plainer words, we define harmful covariate shift based on the existence of multiple \textit{good} models on $\Pc$ that we assume learn the same generalization set, but
    that tend to disagree on $\Qc$ with greater probability compared to $\Pc$.
\end{definition}


\section{Constrained Disagreement Classifiers}\label{sec:constrained-disagreement-classifier-(cdc)}
Our strategy for detecting HCS will be to create an ensemble of \textit{constrained disagreement classifiers} $\{g_1, \ldots, g_\aleph \}$,
classifiers created by the same learning algorithm as $f$ that are constrained to predict consistently (i.e., predict the same as $f$) on $\boldP$ but as differently as possible on $\boldQ$.
If $\Qc$ is within $\Rc$ then such an ensemble will fail to predict differently, as the invariances induced by $L$ and $\boldP$ by definition do not allow a high degree of uncertainty within $\Rc$.
However, when we can find an ensemble that exhibits inconsistent behaviour on $\Qc$, there must be covariate shift that explicitly lies outside $\Rc$.
To make the idea of constrained disagreement classifiers tangible we propose a simple definition which we will translate into a learning algorithm in the following sections.

\begin{definition}
    \label{def:cdc}
    [Constrained Disagreement Classifier (CDC)]
    A constrained disagreement classifier $g_{(f, \boldP, \boldQ)}$, or simply $g$ or $g_\boldQ$ if $f$, $\boldP$ and $\boldQ$ are clear with context, is a classifier with the following properties:
    \begin{enumerate}
        \item $g$ belongs to the same model class as $f$ and is updated with the same learning algorithm when it observes samples from $\Pc$;
        \item $g$ achieves similar training and held out performance on $\Pc$ with respect to $f$;
        \item $g$ disagrees maximally with $f$ on elements of dataset $\boldQ$ while not violating 1 and 2.
    \end{enumerate}
    Our definition of a CDC aims to explicitly capture the concept of a classifier that learns the same generalization region as $f$ while behaving as inconsistently as possible on $\boldQ$.
\end{definition}


\section{Domain Classification and Model Capacity}\label{sec:relationship-with-learning theory}
In our background on covariate shift (\autoref{sec:cov-high-dimensions}) we discussed the mathematical concept of $\mathcal{A}$ distance from~\citep{atheory, domainrep}
and how it translates to the classifier two sample test~\citep{paz2017revisiting}.
Naively, the idea of training classifiers to disagree on out of domain data is identical to the simpler concept of domain classification when given models with sufficiently a large learning capacity e.g., deep neural networks.
However, vanilla domain classification does not leverage the structure and invariance of the specific problem, which will in general result in a less informative test as we have motivated previously (\autoref{sec:harmful-covariate-shift}).
In \autoref{fig:hcs} we explore in a visual toy example how the capacity of CDCs influences the sensitivity for detecting certain distribution shift.

\begin{figure}[!htb]
    \centering
    \includegraphics[width=\linewidth]{images/hcs.pdf}
    \caption{\small Investigating the relation between CDC model complexity and the ability to identify covariate shift.
    We consider a toy example where the ground truth labels are generated using a quadratic decision boundary, shown as a black dashed line.
    The blue points correspond to training samples, and the orange and green points correspond to two different covariate shifts, one closer to the training distribution and the other further.
        (Left) When we choose an underspecified model family (e.g., linear classifiers), there exists no CDC that reports different explanations for the orange and green points.
        However, if expert knowledge led us to believe that a linear classifier is the true causal predictor for the entire domain, we would consequently not be worried about covariate shift.
        (Center) When we choose a quadratic function family, there exists enough variation within the space of models that explain the training set to offer different explanations on the distance shift (orange) but not on the near shift (green).
        An analogy to a more complex learning problem would be that the green region represents a set of datapoints shifted from the source, yet still within the generalization/invariance set of the learning algorithms.
        (Right) When we learn from an overly expressive function family (polynomials of degree 3+), the space of models that explain the training set can offer different explanations of even near covariate shifts (green points).}
    \label{fig:hcs}
\end{figure}

Beyond the concept of shift harmfulness, we will see in our methods and empirical study that unlike domain classifiers, CDCs can easily leverage a pretrained model to result in significantly larger statistical power when detecting shift.


\section{Connection to PQ Learning}\label{sec:relationship-with-pq-learning}

As our work builds on PQ learning, we provide a summary of the original framework and clearly state the distinctions in our methodology.
In PQ learning we seek a selective classifier $\left. h\right|_{S}$ that achieves
a bounded tradeoff between its in distribution rejection rate $\text{rej}_h(\mathbf{x})$ and its out of distribution error $\operatorname{err}_h(\tilde{\mathbf{x}})$ with respect to a ground truth decision function $d$.
Formally, a selective classifier predicts according to a base classifier $h$ if the input $x$ is inside some set $S$ and otherwise rejects/abstains from predicting.
\begin{equation}
    \left. h\right|_{S}(x) \coloneqq \begin{cases}
                                         h(x) & x\in S \\
                                         \text{reject} & x\not\in S
    \end{cases}
    \label{eq:selective_classifier_defn}
\end{equation}
The error and rejection rate of a selective classifier are defined for a selective classifier $\left. h\right|_{S}$ with respect to a distribution $\mathcal{P}$ as:
\begin{equation}
    \text{rej}_{h}(S) \coloneqq \mathbb{P}_{x\sim\mathcal{P}}\left[x\not \in S\right]\quad \text{err}_h(S) \coloneqq \mathbb{P}_{x\sim\mathcal{P}}\left[h(x) \neq y\land x\in S\right]
\end{equation}
Where $y=d(x)$ is the ground truth label for $x$.
The \textit{empirical} rejection and error rates for a set of samples $\mathbf{x}=\{x_i\}_{i=1}^n$ and corresponding labels $\mathbf{y}=\{y_i\}_{i=1}^n$ are similarly defined as:
\begin{equation}
    \text{rej}_{h}(\mathbf{x}) \coloneqq \frac{1}{n}\sum_{i=1}^n \delta_{x_i\not \in S}\quad \text{err}_h(\mathbf{x}) \coloneqq \frac{1}{\sum_{i=1}^n x_i\in S} \sum_{i=1}^n \left(\delta_{x_i\neq y_i} \times \delta_{x_i\in S}\right)
\end{equation}
\noindent
Using this construction \citeauthor{pqlearn} define PQ learning with the following learning theoretic bound.

\begin{definition}[{PQ learning~\citep{pqlearn}}]
    Learner $L$ $(\epsilon, \delta, n)$-PQ-learns $F$ if for any distributions $\Pc, \Qc$ over $X$ and any ground truth function $d \in F$, its output $h\coloneqq L(\boldP, d(\boldP), \boldQ)$ satisfies
    \begin{equation}
        \mathbb{P}_{{\mathbf{x} \sim \Pc^{n}},\ {\tilde{\mathbf{x}} \sim \Qc^{n}}}\left[\text{rej}_h(\mathbf{x})+\operatorname{err}_h(\tilde{\mathbf{x}}) \leq \epsilon\right] \geq 1-\delta
        \label{eq:pqlearn}
    \end{equation}
    $L$ PQ-learns $F$ if $L$ runs in polynomial time and if there is a polynomial $p$ such that $L$ $(\epsilon, \delta, n)-$PQ learns $F$ for every $\epsilon, \delta>0, n \geq p(1 / \epsilon, 1 / \delta)$.
\end{definition}
\citeauthor{pqlearn} propose the Rejectron algorithm for PQ learning in a noiseless binary classification setting with zero training error and access to a perfect empirical risk minimization (ERM) oracle.
Rejectron sequentially searches for new classifiers that achieve zero training error while attempting to predict the opposite label on subsets of the unlabeled set $\boldQ$.
We highlight several pitfalls that prevent a practical realization of Rejectron (1) the classification problem must be binary (2) the optimization objective at each step requires an ERM query with $\Omega(|\boldP|^2)$ samples and (3) most significantly there is no process to control overfitting of large capacity models.
By tackling the above issues we derive a practical PQ learning algorithm and propose a powerful hypothesis test to leverage the PQ learner to identify harmful covariate shift.


\section{Learning to Disagree}\label{sec:dis}
To train a classifier to disagree in the binary setting, it suffices to flip the labels.
However, in the multi-class classification, it is unclear what a good objective function is.
We formulate an explicit loss function that can be minimized via gradient descent to learn a CDC.
For classification problems, letting $g(x_i)$ be the predictive distribution over $N$ classes, $\hat{f}(x_i)\in \{1,\dots N\}$
the label predicted by $f$ and $\mathbb{I}[\cdot ]$ a binary indicator, we define the \textit{disagreement-cross-entropy} (DCE) $\small \tilde{\ell}$ as:
\begin{equation}
    \tilde{\ell}_{\text{DCE}}(g(x_i), f(x_i)) =  \frac{1}{1-N} \sum_{c=1}^N \mathbb{I}[{\hat{f}(x_i) \neq c}] \log(g(x_i)_c)
    \label{eq:anitcross}
\end{equation}
$\tilde{\ell}$ corresponds to taking the cross entropy of $\hat{y}$ with the uniform distribution over all classes except $f(x_i)$.
Since the primary criteria is that $g(x_i)$ disagrees with $f(x_i)$, $\tilde{\ell}$ is designed to minimize the probability of $g$'s prediction for the output of $f$'s while maximizing its overall entropy.
Our definition of $\tilde{\ell}$ is significantly more stable to optimize compared to simply maximizing the regular cross entropy as it has a bounded global minimum.

Our goal is to agree on $\boldP$ and disagree on $\boldQ$.
Consequently, we learn with the loss in \autoref{eq:loss}. $\ell$ denotes the standard cross entropy loss and $\tilde{\ell}$ is the disagreement cross entropy. $\lambda$ is a scalar parameter that controls the trade off between agreement and disagreement.
\begin{equation}
    \mathcal{L}_{\text{CDC}}(\boldP,\boldQ) = \frac{1}{|\boldP|+|\boldQ|}\left(\sum_{(x,y)\in \boldP} \ell(g(x), y) + \lambda \sum_{\tilde{x}\in \boldQ} \tilde{\ell}_{\text{DCE}}(g(\tilde{x}), f(\tilde{x})) \right)
    \label{eq:loss}
\end{equation}
When learning CDCs in practice, $\mathcal{L}_\text{CDC}$ should be combined with any additional regularization and data augmentation used in the original training process of $f$ to ensure that we retain the true generalization region of $f$.
Furthermore, training and validation metrics must be closely monitored on unseen samples from $\Pc$ to ensure that $g$ achieves similar generalization performance on $\Pc$.

\subsection{Choosing $\lambda$}
%In the original formulation of Rejectron, selective classifiers are trained on a dataset consisting of $\boldP$ replicated $|\boldP|$ times and $\boldQ$.
%Calling an ERM oracle on this data ensures that a misclassification on $\boldP$ is significantly more costly than one on $\boldQ$ but requires $\Omega(|\boldP|^2 )$ samples, an impractical number for large datasets.
We show that we can choose the scalar parameter $\lambda$ in \autoref{eq:loss} to set learning $\boldP$ as the primary learning objective and only when it cannot be improved, we allow $g$ to learn how to disagree on $\boldQ$,
thus relaxing the original constrained optimization objective to a tractable weighted objective.
The reasoning is a simple counting argument.
Suppose agreeing on each sample in $\boldP$ incurs a reward of $1$ and disagreeing with each sample in $\boldQ$ a reward of $\lambda$.
To encourage agreement on $\boldP$ as the primary objective, we set $\lambda$ such that the extra reward obtained by going from \textit{zero} to \textit{all} disagreements on $\boldQ$ is less than that achieved with only one extra agreement on $\boldP$, this gives $\lambda|\boldQ| < 1$.
Practically, we chose $\lambda=1/(|\boldQ| + 1)$ and find that no tuning is required.

\subsection{Generalizing Beyond Cross Entropy}\label{subsec:gen_beyond}     When training models with arbitrary discrete or generally non-differentiable with respect their objective (e.g., random forest), we must find a more general solution for creating CDCs.
Such a solution should (1) reduce to the DCE when the model is, in fact, continuous and trained using the standard cross-entropy, and (2) reduces to label flipping when $N=2$ (binary classification).
Our simple solution is to replicate every sample in $\boldQ$ exactly $N-1$ times and create a unique label for each
from the set $\mathcal{S}\coloneqq \{1,\dots,N\}\backslash \{t\}$ where $t$ is the disagreement target.
We also give each a sample a weight of $1/(N-1)$.
In the case of $N=2$, this corresponds to no replication and simply assigning the opposite label.
In the case where the model learns by cross-entropy, this generalization corresponds to the DCE in \autoref{eq:anitcross}; we provide a simple proof below.
\smallbreak\noindent
\begin{proof}
    Starting with the definition of the cross entropy
    \begin{equation}
        \text{CE}(f(x), y) = -\sum_{c=1}^N \mathbb{I}[{c=y}] \log(f(x)_c)
    \end{equation}
    Now we consider the sum of the cross entropy for each label in $\mathcal{S}$:
    \begin{align}
        \sum_{y\in \mathcal{S}} \text{CE}(f(x), y) &= -\sum_{y\in \mathcal{S}} \sum_{c=1}^{N}\mathbb{I}[{c=y}]\log(f(x)_c) \\
        &=-\sum_{c=1}^{N}\mathbb{I}[{c\in \mathcal{S}}]\log(f(x)_c)\\
        &= (N-1) \tilde{\ell}_\text{DCE}(f(x), y)
    \end{align}
    Hence when giving each sample a weight of $1/(N-1)$ we recover the exact from of DCE.
\end{proof}

\subsection{Ensembling} To learn richer disagreement rules, we create an
ensemble of CDCs where the $k^{\text{th}}$ model is trained only to disagree on the subset of
$\boldQ$ that has yet to be disagreed on by models $1$ through $k-1$.
The final disagreement rate $\phi_\boldQ$ is the fraction of unlabelled samples where any CDC provides an alternate decision from $f$.
In what follows we use this rate to detect shift.

\begin{algorithm}
    \DontPrintSemicolon
    \SetAlgoLined
    \SetNoFillComment
    \SetKwRepeat{Do}{do}{while}%
    \caption{Constrained Disagreement}\label{alg:constrained}
    \KwIn{$L$: learning algorithm, $\boldP_{\text{train}}$: labeled training dataset $\{\dots,(x_i,y_i),\dots\}$,
        $\boldP_{\text{val}}$: labeled validation dataset $\{\dots,(x_i,y_i),\dots\}$, $\boldQ$: unlabeled test dataset $\{\dots,{x}_i,\dots\}$,
        $f$: classifier trained on $\boldP$, $\mathcal{M}$: evaluation metric (default \textit{accuracy}), $\varepsilon$: tolerance (default 0.05),
        $k$: max epochs (default 10)}
    \KwOut{Constrained Disagreement Classifier $g_{(f,\boldP, \boldQ)}$}
    $\hat{\boldQ}\ \gets\ \{({x}, f({x}))\ |\ \hat{x}\in \boldQ\}$ \tcp*{infer pseudo labels on $\boldQ$ using $f$}
    \tcp*{create a batched dataloader using $\boldP \text{ and } \boldQ$}
    ${\boldP\boldQ}\ \gets\ $Batched($\{(x,y)\ |\ (x,y)\in \boldP \land (x,y)\in \boldQ$\})\\
    $g\ \gets\ f$\tcp*{Initialize $g$ with $f$}
    $m_0\ \gets\ \mathcal{M}(f, \boldP_{\text{val}})$ \tcp*{Compute the validation performance of $f$}
    \While{$\mathcal{M}(f, \boldP_{\text{val}}) > m_0 - \varepsilon$ and iterations $<\ k$}{
        \tcp*{Training epoch over ${\boldP\boldQ}$}
        \For{batch $\ \in \boldP\boldQ$}{
            $x_P,\ y_P\ \gets\ \{(x, y)\ |\ (x,y)\in\text{ batch}\text{ and }(x,y)\in {\boldP}_{\text{train}}\}$\\
            $x_Q,\ y_Q\ \gets\ \{(x, y)\ |\ (x,y)\in\text{ batch}\text{ and }(x,y)\in \hat{\boldQ}\}$\\
            \tcp*{Update $g$ to minimize the loss in \autoref{eq:loss} and base loss $\ell$ defined by $L$}
            $g\ \gets \text{Update}(g, L, (x_P, y_P), (x_Q, y_Q))$
        }
    }
    \Return{$g$}
\end{algorithm}


\section{Detecting Shift with Constrained Disagreement}\label{sec:from-constrained-disagreement-to-detecting-shift-with-hypothesis-tests}
A natural way to apply the concept of constrained disagreement to the identification of covariate shift is
to partition $\boldQ$ into two sets, using the first to train a CDC ensemble and the second to compute an unbiased estimate
of its held out disagreement rate $\phi_\Qc$.
We would statistically compare this disagreement rate using a $2\times 2$ exact hypothesis test (e.g., Fisher's or Barnard's) against a baseline estimate for the disagreement rate on $\Pc$ computed using unseen data.
The following shows that this results in a provably correct method to detect shift with high probability using only finite samples.
\begin{theorem}[Disagreement implies covariate shift]
    \label{th:dis}
    Let $f$ be a classifier trained on dataset $\boldP$ drawn from $\Pc$ and their corresponding labels.
    Let $g$ be a classifier that is observed to agree (classify identically) with $f$ on $\boldP$ and disagree on a dataset $\boldQ$ drawn from $\Qc$.
    If the rate which $g$ disagrees with $f$ on $n$ unseen samples from $\Qc$ is greater than that from $n$ unseen samples from $\Pc$ with probability greater than $p^\star \coloneqq \frac{1}{2}\left(1- 4^{-n} \binom{2n}{n}\right)$ there must be covariate shift.
\end{theorem}
\textit{Sketch of Proof}.
We show that under the null hypothesis where $\Pc = \Qc$ the tightest upper bound on the probability that $g$ is more likely to disagree on $\Qc$ compared to $\Pc$ is $p^\star$.
The contrapositive argument then states
if we deem the probability to be greater than $p^\star$ there must be a covariate shift.
This result motivates a hypothesis testing approach to determine how probable it is that $g$ is truly more likely to disagree on $\Qc$ given only a set of finite observations.
The full proof can be found in \autoref{sec:proofs}.

\textit{Remark}.
While \autoref{th:dis} takes a frequentist's approach to identifying covariate shift, we show in \autoref{sec:proofs} (Theorem 2) that there is a Bayesian formulation which can provide a closed-form and informative lower bound on the probability
of covariate shift given an observation of finite sample disagreement rates on $\boldP$ and $\boldQ$.

Our theory, while simple, has a limitation that prevents its direct application.
Any approach that requires unseen samples from $\boldQ$ is ill-suited for the low data regime, as it requires splitting $\boldQ$ leaving an even smaller set
for computing the disagreement rate.
Estimators from small samples result in high variance and ultimately low statistical power.
Furthermore for small samples sizes the ability to find the classifier $g$ that meets of predicate of \autoref{th:dis} becomes incredibly difficult.
Since our objective is to detect covariate shift from as few test samples as possible, splitting $\boldQ$ and computing disagreement on unseen data will not result in a powerful test.
To tackle this issue practically, we take a transductive approach based on intuition from learning theory and observed dynamics of CDC learning.
We claim that training learning models to disagree on samples from $\boldQ$ while generalizing to $\Rc$ is a \textit{far easier} task when $\Qc$ is not in $\Rc$.
Easiness in this setting is captured by the fact that standard learning algorithms tend to converge significantly quicker to the CDC objective when $\Pc \neq \Qc$.
We can therefore use the \emph{relative increase} in disagreement between CDCs \textit{during training} on $\boldQ$ and $\boldP$ to capture a quantity that is more informative than the unbiased statistic without reducing samples from $\boldQ$ that we can use.


\section{The Detectron Test}\label{sec:the-detectron-test}

\subsection{Detectron Disagreement}
Our proposed method is to train two CDC ensembles of size $\aleph$
denoted by $\mathbf{g}_{\boldQ}\coloneqq \{g_{\boldQ}^1,\ldots {g}_{\boldQ}^\aleph\}$ and $\mathbf{g}_{\Pun}\coloneqq \{g_{\Pun}^1,\ldots g_{\Pun}^\aleph\}$.
First, $g_{\boldQ}$ is trained to disagree with $f$ on $\boldQ$, while constrained to agree on the original training set $\boldP$.
Next, enforcing the null hypothesis, we train $\mathbf{g}_{\Pun}$ to disagree on set of \textbf{unseen samples} $\Pun$ drawn from $\Pc$ where $m\coloneqq|\Pun|=|\boldQ|$.
To maximize sample efficiency, we take a transductive approach to detecting shift by analyzing the outputs of $\mathbf{g}_{\boldQ}$ and $\mathbf{g}_{\boldP}$ on the sets $\boldQ$ and $\Pun$ themselves by
calculating their \textit{empirical disagreement} during training.

We define the empirical disagreement between a base classifier $f$, an ensemble of classifiers $\mathbf{g}\coloneqq \{g^1,\dots,g^\aleph \}$, and an unlabeled dataset $\mathbf{D}\coloneqq \{x_i\}_{i=1}^n$ as the proportion of samples in $\mathbf{D}$
where any $g\in \mathbf{g}$ predicts differently from $f$:
\begin{equation}
    \phi_{\mathbf{D}}(f,\mathbf{g}) \coloneqq \frac{1}{|\mathbf{D}|} \sum_{x\in \mathbf{D}} \left(1-\mathbb{I}\left[\bigwedge\limits_{g\in \mathbf{g}} \hat{f}(x)=\hat{g}(x)\right]\right)
\end{equation}
The fraction of samples that $\mathbf{g}_{\Pun}$ disagrees on with respect to $f$ on $\Pun$ is denoted using shorthands $\phi_{\Pun}\coloneqq \phi_{\Pun}(f, \mathbf{g}_{\Pun})$,
and similarly for $g_{\boldQ}$ as $\phi_{\boldQ}$.
Note that $\phi_{\Pun}$ and $\phi_{\boldQ}$ are random variables that obey intractable distributions governed by the dynamics of the CDC learning algorithm and disagreement sets $\Pun$ and $\boldQ$.
The observed low variance and large divergence between the distributions of $\phi_{\Pun}$ and $\phi_{\boldQ}$ under covariate shift forms the backbone the \method test.
Under the null hypothesis $\mathcal{H}_0$, if $\Qc$ is not a harmful shift from $\Pc$,
then the intuition we have built suggests that $\phi_{\boldQ}$ will not grow more rapidly than $\phi_{\Pun}$
\begin{equation}
    \mathcal{H}_0: \mathbb{E}[\phi_\boldQ] \leq \mathbb{E}[\phi_\Pun], \quad  \mathcal{H}_a: \mathbb{E}[\phi_\boldQ] > \mathbb{E}[\phi_\Pun]
    \label{eq:harmfulshifttest}
\end{equation}
The converse of harmful shift is expressed as the one-sided alternative $\mathcal{H}_a$, meaning disagreeing on $\boldQ$ \textit{is easier} than $\Pun$.
See \autoref{fig:harmfulshift} for a visual depiction of CDC training dynamics when applied to the near distribution shift benchmark CIFAR 10/10.1~\citep{cifar101}.

\begin{figure}[!htb]
    \floatbox[{\capbeside\thisfloatsetup{capbesideposition={right,top},capbesidewidth=0.4\linewidth}}]{figure}[\FBwidth]
    {\caption{\small \textbf{CDC Training Dynamics:} In blue we train CDCs to disagree on a set of 100 samples from CIFAR 10.1~\citep{cifar101} ($\boldQ$) -- a near OOD test set for CIFAR 10 -- while in black
    we force CDCs to disagree on the original CIFAR 10 test set ($\Pun$).
    Even after a small number of training batches the empirical disagreement rate grows significantly larger CIFAR 10.1 compared to CIFAR 10.
    The observation that CIFAR 10.1 results in more predictive variance under the CDC algorithm suggest that this covariate shift is harmful}
    \label{fig:harmfulshift}}
    {\includegraphics[width=\linewidth]{images/training_cdcs_small.pdf}}
\end{figure}
We refer to the test in \autoref{eq:harmfulshifttest} as \textit{\method\ Disagreement} (Det-D).
To compute the test result at a significance level $\alpha$ we first estimate the null distribution of $\phi_\boldP$ for a fixed sample size $n$ by training $K$ calibration rounds of $g_{\boldP}$ with different random sets $\Pun$ of size $n$.
The test result is significant if the observed disagreement rate $\phi_{\boldQ}$ is greater than the $(1-\alpha)$ quantile of the null distribution.
This process is explained visually in \autoref{fig:detectron_permutation}.

\begin{figure}[!htb]
    \centering
    \includegraphics[width=\linewidth]{images/detectron_permutation.pdf}
    \caption{\small \textbf{The \method\ Disagreement Test:} (taken from our experiment where $\Pc\ =\ $CIFAR 10 and $\Qc\ =\ $CIFAR 10.1 and sample size $n=50$) We start by
    training an ensemble of CDCs (we use and ensemble size of $5$) to disagree on a set of $n$ unseen samples $\Pun$ from the original training distribution
    while constrained to perform consistently with a base model on the original training and validation sets used to train the base model on CIFAR 10.
    We perform 100 of these calibration runs using different random seeds and samples for $\Pun$ to estimate a threshold $\tau$ such that $95\%$
        of the runs disagree on fewer than $\tau$ samples --- thereby fixing the significance level of the test to $5\%$.
        To estimate the test power, we train CDCs using \textbf{the exact same configuration} as the calibration runs except we replace $\Pun$ with a random set of $n$ samples $\boldQ$ from $\Qc$ (CIFAR 10.1).
        By averaging the number of runs that disagree on more than $\tau$ samples from $\boldQ$ we can compute the power (or true positive rate) of the test as $83\%$.}
    \label{fig:detectron_permutation}
\end{figure}

\subsection{Detectron Entropy}
We consider an additional variant, \textit{\method\ Entropy} (Det-E), which computes the mean prediction entropy of the CDC instead of relying solely on disagreement rates.
The intuition for Det-E draws from the fact that when CDCs satisfy their objective (i.e., in the case of harmful shift) they
learn to predict with high entropy on $\boldQ$ and low entropy on $\Pun$, resulting in a natural way to distinguish between distributions.
The CDC entropy is computed for a sample $x$ from the mean probabilities over each $N$ classes of the base classifier $f$ and set of CDCs $\mathbf{g}$.
\begin{equation}
    \text{CDC}_{\text{entropy}}(x)=-\sum_{c=1}^N {\hat{p}_c\log(\hat{p}_c)}\ \text{ where }\ \hat{p}_c \coloneqq \frac{1}{|\mathbf{g}|+1}\left(f(x)_c + \sum_{g\in \mathbf{g}} g(x)_c \right)
    \label{eq:cdc_entropy_def}
\end{equation}
We use a KS test to compute a $p$-value for covariate shift directly on the entropy distributions computed for $\boldQ$ and $\Pun$ and guarantee significance using the same strategy as Det-D
(This process is explained visually in \autoref{fig:entropy_test}).
The intuition for \textit{\method\ (Entropy)} draws from the fact that when CDCs satisfy their objective (i.e., in the case of harmful shift) they
learn to predict with high entropy on $\boldQ$ and low entropy on $\Pun$, resulting in a natural way to distinguish between distributions.

\begin{figure}[!htb]
    \centering
    \includegraphics[width=\linewidth]{images/entropy_test.pdf}
    \caption{\small \textbf{The \method\ Entropy Test:} Following the same experimental setup as \autoref{fig:detectron_permutation},
        we start (left) by computing a KS test between the empirical distributions entropy values for each calibration run $\Pun$ with the flattened set of entropy values from all other calibration runs.
        Then (center) we compute a KS test from each test run $\boldQ$ with a random set of all but one calibration runs.
        Finally (right), we find a threshold $\tau$ on the distribution of $p$-values obtained from step 1 as the lower $\alpha$ quantile to guarantee a false positive rate of $\alpha$.
        The power of the test is computed as the fraction of $p$-values computed from the test runs on $\boldQ$ that are below $\tau$.}
    \label{fig:entropy_test}
\end{figure}

\clearpage
\subsection{The Detectron Algorithm}
Concluding our methodology, we present a formal algorithmic description of the Detectron Disagreement test below in \autoref{alg:detectron}.
We release our code as open source software under a GNU General Public License v3.0.
\vspace{-5mm}
\begin{center}
    \href{https://github.com/tomginsberg/deeptst}{\includegraphics[width=.4\linewidth]{images/logo.pdf}}\\
    \vspace{-8mm}
    \url{https://github.com/tomginsberg/deeptst}
\end{center}

\begin{algorithm}[!htbt]
    \DontPrintSemicolon
    \SetAlgoLined
    \SetNoFillComment
    \SetKwRepeat{Do}{do}{while}%
    \caption{The Detectron algorithm for detecting harmful covariate shift}\label{alg:detectron}
    \KwIn{$\boldP$: labeled dataset, $\boldQ$: unlabeled dataset, ${L}$: learning algorithm,\\
        $K$: calibration rounds $=100$, $\aleph$: ensemble size $=5$, $\alpha$: significance level $=0.05$}
    \KwOut{test result for harmful covariate shift at significance level $\alpha$}
    \BlankLine
    $\boldP_{\text{train}},\ \boldP_{\text{val}},\ \Pun\ \gets\ \text{Partition}(\boldP)$

    $m\ \gets\ |\boldQ|$;\ $\Phi_{\Pun} \gets [\ ]$

    $f\ \gets\ {L}(\boldP_{\text{train}},\ \boldP_{\text{val}})$\ \tcp*{ Load or train a base classifier on $\boldP$}

    \Repeat{K iterations elapse}{
        $\mathbf{p}^\star\ \gets\ \text{ RandomSample}(\Pun, m)$

        \tcp{ Train an ensemble of CDCs on $\Pun$}
        \While{$|\mathbf{p}^\star|>0$ and iterations $\leq \aleph$}{
            $g\ \gets\ \text{ConstrainedDisagreement}({L}, \boldP_{\text{train}}, \boldP_{\text{val}}, \mathbf{p}^\star, f)$ \tcp*{See \autoref{alg:constrained}}

            $\mathbf{p}^\star \ \gets\ \{x\ |\ x\in \mathbf{p}^\star\text{ and } f(x)=g(x)\}$\tcp*{Filter out disagreed on data}

            $\phi_\Pun \ \gets 1-|\mathbf{p}^\star|/m$
            \tcp*{Update disagreement rate}}

        $\text{Append}\ \phi_{\boldP}\text{ to }\Phi_\boldP$}
    \tcp{ Train an ensemble of CDCs on $\boldQ$}
    \While{$\boldQ>0$ and iterations $\leq \aleph$}{
        $g\ \gets\ \text{ConstrainedDisagreement}({L}, \boldP_{\text{train}}, \boldP_{\text{val}}, \boldQ, f)$

        $\boldQ\ \gets\ \{x\ |\ x\in \boldQ\text{ and } f(x)=g(x)\}$

        $\phi_\boldQ \ \gets 1-|\boldQ|/m$
    }
    \Return{$\phi_\boldQ > \left[{(1-\alpha)}\text{ quantile of }\Phi_\Pun \right]$}
\end{algorithm}


    \chapter{Applications and Experiments}\label{ch:experiments}
    \section{Datasets}
Our experiments are carried out on natural distribution shifts across multiple domains, modalities, and model types.
We use the \textit{CIFAR-10.1} dataset~\citep{cifar101} where shift comes from subtle changes in the dataset creation processes,
the \textit{Camelyon17 dataset}~\citep{camelyon} for metastases detection in histopathological slides from multiple hospitals, as well as the \textit{UCI heart disease} dataset~\citep{misc_heart_disease_45} which contains tabular features collected across international health systems and indicators of heart disease.

A summary of the three datasets used as well as the description of shifts and their impacts on base model performance is provided in \autoref{tab:datasets_main} below.
\begin{table}[!htb]
    \centering
    \setlength\tabcolsep{1.5pt}
    \renewcommand{\arraystretch}{0.5}
    \small
    \caption{\small \textbf{Datasets:} We investigate three different forms of covariate shift in two unique data modalities.
    To verify that these shifts are indeed harmful to the models, we report performance in both the shifted and unshifted domains.
    Examples and further descriptions of unshifted/shifted splits of each dataset are given in \autoref{sec:data}.}
    \resizebox{\textwidth}{!}{
        \begin{tabular}{cccccc}
            \toprule[1.5pt]
            Domain/Task & {Dataset} & \thead{Shift} & \thead{Metric} & \thead{(Unshifted)} & \thead{(Shifted)} \\\midrule[1.5pt]
            \thead{Natural Images \\\emph{Object classification}} & \thead{CIFAR-10/10.1 \\~\citep{cifar101}} & \thead{Data Collection \\ Process} & \thead{Accuracy} & \thead{0.87 (Resent18)} & \thead{0.77 (Resent18)} \\\midrule
            \thead{Histopathological Images \\ \emph{Metastases Detection}} & \thead{Camelyon-17 \\\citep{camelyon}} & \thead{Different \\Hospitals} & \thead{Accuracy} & \thead{0.93 (Resent18)} & \thead{0.81 (Resent18)} \\\midrule
            \thead{Tabular Medical Data \\\emph{Angiographic Status}} & \thead{UCI Heart Disease \\\citep{misc_heart_disease_45}} & \thead{Different \\Countries} & \thead{AUROC} & \thead{0.88 (xgboost)\\0.85 (MLP)} & \thead{0.70 (xgboost)\\0.42 (MLP)} \\\midrule
        \end{tabular}}
    \label{tab:datasets_main}
\end{table}


\section{Learning Constrained Disagreement}
Before discussing the results of the Detectron test, we begin by investigating the CDC algorithm (\autoref{alg:constrained}) in isolation to verify that it meets our desired criteria in \autoref{def:cdc}.
We begin by training ensembles of $10$ CDCs using the \textit{disagreement cross entropy} (DCE) with CIFAR-10 as $\Pc$ and CIFAR-10.1 as $\Qc$ for 100 random runs at a sample size of $50$.
We use an ImageNet pretrained residual network fine-tuned on CIFAR 10 as our basemodel $f$; see \autoref{sec:expdet} for full training details.
CDCs are initialized from $f$ and use the same training algorithm/loss function when updated on samples from $\Pc$ (property 1 in \autoref{def:cdc}).
The results in \autoref{fig:disrates} empirically validates minimizing the DCE as a CDC learning objective.
The first observation is that when an unseen set is drawn from a shifted distribution $\Qc$, the empirical disagreement rate $\phi_\boldQ$ grows significantly larger
than the baseline disagreement rate $\phi_\boldP$ (property 3).
Next, we see that CDCs preserve accuracy on unseen data from the training distribution, regardless of the dataset they are trained to disagree on (property 2).
Finally, an additional observation shows that as the ensemble size increases (and disagreed upon points are removed) we see that the selective classification accuracy on the agreed upon points increases.
Furthermore, the concave behaviour of the selective classification curve indicates that the points that CDCs quickly disagree on many data samples would have been misclassified by $f$,
while in later iterations fewer of such samples are identified (see \autoref{fig:selective_cls} for a direct comparison of accuracy vs rejection rate of CDCs).

\begin{figure}[!htb]
    \centering
    \includegraphics[width=\textwidth]{images/acc_dis.pdf}
    \caption{\small \textbf{Ensemble Size vs Properties of Constrained disagreement classifiers on CIFAR-10/10.1:}
    (Left) We see that for all ensemble sizes, there is lower disagreement on unshifted data (CIFAR-10) compared to disagreement on shifted data (CIFAR-10.1).
        (Center) Constrained disagreement does not compromise in-distribution performance as computed on unseen data.
        (Right) As the ensemble grows the selective classification accuracy computed on the set of test examples that all models agree on increases both on in-distribution and out-of-distribution data.
        Confidence intervals are computed as $\pm$ one standard deviation across experiments.}
    \label{fig:disrates}
\end{figure}


\begin{figure}[!htb]
    \floatbox[{\capbeside\thisfloatsetup{capbesideposition={right,top},capbesidewidth=0.4\linewidth}}]{figure}[\FBwidth]
    {\caption{\small \textbf{Selective Classification Behaviour of CDCS}: Comparing the accuracy versus rejection rate of selective classification on ID (CIFAR 10) and shifted (CIFAR 10.1)
        sets we see that the CDC algorithm constructs highly linear selective classifiers. As expected the rejection rate required to reach high accuracy on ID data is significantly lower
        than that required on shifted data.}
    \label{fig:selective_cls}}
    {\includegraphics[width=\linewidth]{images/selective_classification}}
\end{figure}


\section{Shift Detection Approach} We evaluate the \method\ in a standard two-sample testing scenario similar to prior work~\citep{zhao2022comparing}.
Given two datasets $\boldP = \{(x_i,y_i)\}_{i=1}^n$ ($x_i$ drawn from $\Pc$) and $\boldQ = \{\tilde{x}_i\}_{i=1}^m$ ($\tilde{x}_i$ drawn from $\Qc$) and classifier $f$,
we seek to rule out the null hypothesis ($\Pc = \Qc$) at the $5\%$ significance level.
To guarantee fixed significance we employ a permutation test by first sampling from the distribution of test statistics
derived by the \method\ where the null hypothesis $\Pc=\Qc$ holds (i.e., $\boldQ$ is drawn $\Pc$).
Next, we compute a threshold over the empirical test statistic distribution that sets the false positive rate to $5\%$ (see \autoref{fig:detectron_permutation}).
This step can be performed before deployment as it only requires access to $\boldP$.
To mimic deployment settings where we wish to identify covariate shift quickly,
we assume access to far more samples from $\Pc$ compared to $\Qc$.
For each dataset, we begin by training a base classifier on the unshifted dataset.
We evaluate the detection of covariate shift on 100 randomly selected test sets of $n=$ 10, 20 and 50 samples from $\Qc$.
In all cases, we train a maximum ensemble size of 5 (parameter $\aleph$ in \autoref{alg:detectron}).
To prevent CDCs from overfitting in the case of small test set sizes, we perform early stopping if in-distribution validation performance drops by over 5\% from the measured performance of the base classifier.
Hyperparameters and training details for all models can be found in \autoref{sec:expdet}.

\subsection{Evaluation} We report the \textit{True Positive Rate at 5\% Significance Level (TPR@5)} aggregated over $100$ randomly selected sets $\boldQ$.
This signifies how often our method correctly identifies covariate shift ($\Pc\neq\Qc$) while only incorrectly identifying shift 5\% of the time.
This is also referred to as the statistical power of a test where the significance level or probability of making a type I error is 5\%.
% (2) \textit{Area Under the Receiver Operating Characteristic Curve (AUC)}: To showcase the sensitivity of our method - a desirable characteristic in high-risk domains where false negatives are far more costly than false positives - we report the area under the true positive vs false positive curve generated by varying the significance level from 0 to 1. %This is identical to the conventional interpretation of the AUC for classification problems.

\subsection{Baselines} We compare the \method\ against several methods for OOD detection, uncertainty estimation and covariate shift detection. \textit{Deep Ensembles}~\cite{trustuncert} using both (1) \textit{disagreement} and (2) \textit{entropy} scoring methods as a direct ablation to the CDC approach (3) \textit{Black Box Shift Detection (BBSD)}~\citep{bbsd}.
(4) \textit{Relative Mahalanobis Distance (RMD)}~\citep{relmahala}.
(5) \textit{Classifier Two Sample Test (CTST)}~\citep{paz2017revisiting}.
(6) \textit{Deep Kernel MMD (MMD-D)}~\citep{liu2020learning}.
(7) \textit{H Divergence (H-Div)}~\citep{zhao2022comparing}.
For more information on baselines see Appendix \autoref{subsec:baselines}.


\section{Shift Detection Experiments}
We present statistical power (TPR@5) results for sample sizes of 10, 20 and 50 using the \method (\autoref{alg:detectron})
on all datasets are shown in \autoref{tab:results}.
We report the mean and standard error of TPR@5 computed on 100 samples for each trial.

\begin{table}[!ht]
    \centering
    \caption{\small \textbf{Results (true positive rate at the 5\% significance level) for detection of harmful covariate shift} on CIFAR-10.1, Camelyon 17 and UCI Heart Disease benchmarks.
    The \textbf{best} result for each column is bolded, results that are within $\underline{\text{2\% of the best}}$ are underlined and the \textit{best baseline} method is italicized.}
    \vspace{5mm}
    \setlength{\tabcolsep}{5pt}
    {\renewcommand{\arraystretch}{2}%
    \resizebox{\textwidth}{!}{
        \begin{tabular}{r|ccc|ccc|ccc}
            \toprule[1.5pt]
            \multicolumn{1}{r|}{} & \multicolumn{3}{c|}{\textbf{\large CIFAR 10.1}} & \multicolumn{3}{c|}{\textbf{\large Camelyon 17}} & \multicolumn{3}{c}{\textbf{\large {UCI Heart Disease}}} \\
            \multicolumn{1}{r|}{\large $|\boldQ|$} & 10            & 20            & \multicolumn{1}{c|}{50} & 10                     & 20                     & \multicolumn{1}{c|}{50}   & 10   & 20   & \multicolumn{1}{c}{50}  \\\midrule[1.5pt]
            \thead{ BBSD}                          & $.07\pm.03$   & $.05 \pm .02$ & $.12 \pm .03$           & $.16 \pm .04$          & $.38 \pm .05$ & $.87 \pm .03$ & $.13 \pm .03$ & $.22 \pm .04$  & $.46 \pm .05$  \\
            \thead{ Rel. Mahalanobis}              & $.05 \pm .02$ & $.03 \pm .03$ & $.04 \pm .02$           & $.16 \pm .04$ & $.40 \pm .05$ & $\mathit{.89 \pm .03}$  & $.11 \pm .03$   & $.36 \pm .05$    & $.66 \pm .05$                            \\
            \thead{ Deep Ensemble\\(Dis)} & $.23 \pm .04$ & $.40 \pm .05$ & $\mathit{.74 \pm .04}$ & $.10 \pm .03$ & $.11 \pm .03$ & $.13 \pm .03$ & $.02 \pm .01$ & $.00 \pm .00$ & $.32 \pm .05$ \\
            \thead{ Deep Ensemble\\ (Entropy)}                         & $\mathit{.33 \pm .05}$ & $\mathit{.52 \pm .05}$ & $.68 \pm .05$           & $.14 \pm .03$ & $.26 \pm .04$ & $.82 \pm .04$ & $.13 \pm .03$   & $.32 \pm .05$   & $.64 \pm .05$                       \\
            \thead{CTST}                         & $.03 \pm .02$  & $.04 \pm .02$  & $.04 \pm .02$            & $.11 \pm .03$           & $.59 \pm .05$           & $.59 \pm .05$   & $\mathit{.15 \pm .04}$ & $\mathit{.51 \pm .05}$ & $\underline{\mathit{.98 \pm .01}}$                         \\
            \thead{ MMD-D} & $.24 \pm .04$ & $.10 \pm .03$ & $.05 \pm .02$ & $\mathit{.42 \pm .05}$ & $\mathit{.62 \pm .05}$ & $.69 \pm .05$ & $.09 \pm .03$ & $.12 \pm .03$ & $.27 \pm .04$ \\
            \thead{ H-Div} & $.02\pm .01$ & $.05\pm .02$ & $.04\pm .02$ & $.03\pm .02$ & $.07\pm .03$ & $.23\pm .04$ & $\mathit{.15 \pm .04}$ & $.26 \pm .04$ & $.37 \pm .05$ \\\midrule[1.5pt]
            \textbf{\thead{ Detectron \\ (Dis)}} & $\mathbf{.37 \pm .05}$ & $\underline{.54 \pm .05}$ & $.83 \pm .04$ & $\mathbf{.97 \pm .02}$ & $\mathbf{1.0 \pm .00}$ & $.96 \pm .02$ & $.24 \pm 0.04$ & $.57 \pm 0.05$ & $.82 \pm 0.04$ \\
            \textbf{\thead{ Detectron \\ (Entropy)}} & $\underline{.35 \pm .05}$ & $\mathbf{.56 \pm .05}$ & $\mathbf{.92 \pm .03}$ & $\mathbf{.97 \pm .02}$ & $\mathbf{1.0 \pm .00}$ & $\mathbf{1.0 \pm .00}$ & $\mathbf{.45 \pm .05}$ & $\mathbf{.88 \pm 0.03}$ & $\mathbf{1.0\pm .00}$ \\
        \end{tabular}
    }}
    \label{tab:results}
\end{table}

To expand on \autoref{tab:results} we show an extended analysis of the performance of the \method\ on the UCI Heart Disease dataset.
Using a sample size ranging from 10 to 100 we compute the TPR@5 averaged over 100 trials and plot the results in \autoref{fig:uci_plot}.
As the \method\ is model agnostic we use gradient boosted trees (XGBoost~\citep{xgb}) for \method\ and CTST
methods while the remaining baselines which require neural network models use a 2 layer MLP that achieves similar in-distribution performance.

\begin{figure}[!htb]
    \centering
    \includegraphics[width=\textwidth]{images/uci_plot.pdf}
    \caption{\small \textbf{True positive rate at the 5\% significance level} for the \method\ and baseline methods for detection of covariate shift on the UCI heart disease dataset.
    The \method\ (Entropy) is shown to uniformly outperform baselines.
    Confidence intervals are excluded for visual clarity but are found in \autoref{tab:results}.}
    \label{fig:uci_plot}
\end{figure}




    \chapter{Discussion}\label{ch:discussion}
    \textit{Overall Performance}.
We observe in the bottom rows of \autoref{tab:results} that \method\ methods outperform all baselines across all three tasks.
This confirms our intuition that designing distribution tests based specifically on available data and the outputs of learning algorithms is a promising avenue for improving existing methods in the high dimensional/low data regime.

\textit{Sample Efficiency}.
For more significant shifts (Camelyon and UCI), we see in \autoref{tab:results} the most significant improvements over baselines in the lowest sample regime (10 data points).
The fine-grained result in \autoref{fig:uci_plot} shows that CTST catches up to \method\ at $40$ samples while deep ensemble, BBSD, and Mahalanobis catch up at $100$.

\textit{Disagreement vs Entropy}.
For the experiments on imaging datasets with deep neural networks \method\ (Disagreement) often performs nearly as well as \method\ (Entropy),
while \method\ (Entropy) is strictly superior for the UCI dataset.
While we recommend entropy as the method to maximize test power, disagreement is a more interpretable statistic as it is correlates well with the portion of misclassified samples (see (right) \autoref{fig:disrates}).

\textit{Comparison to baselines}.
Amongst the baselines, there is no clear best method. Although on average, ensemble entropy is superior on CIFAR, MMD-D on Camelyon, and CTST on UCI. Our method may be thought of as a combination of ensembles, CTST, and H-Divergence. As ensembles, we leverage the variation in outputs between a set of classifiers; as CTST, we learn in a domain adversarial setting; and as H-Divergence, we compute a test statistic based on data that a model was trained on.
Lastly, while MMD-D and H-Divergence were shown to be the previous state-of-the-art, their performance was validated only on larger sample sizes ($\geq$ 200).

\textit{On Tabular Data}.
The \method\ shows promise for deployment on tabular datasets (bottom right of\ \autoref{tab:results} and \autoref{fig:uci_plot}),
where (1) the computational cost of training models is low, (2) the model agnostic nature of the \method\ is beneficial as random forests often outperform neural networks in tabular data~\citep{tabdatasurvey}, and
(3) based on our discussions with medical professionals, the ability to detect covariate shift from small test sizes is of particular interest in the healthcare domain
where population shift is a constant problem burden for maintaining the reliability of deployed models.

\emph{On computational cost: } Our method is more computationally expensive than some existing methods for detecting shifts such as BBSD and Mahalanobis Scores, but is similar complexity to other approaches such as Ensembles, MMD-D and H-Divergence which may require training multiple deep models.
However, as the \method\ leverages a pretrained model already in deployment, we find in practice that only a small number of training rounds are required to create each CDC.
For instance, on CIFAR 10/10.1 a CDC ensemble of 5 models using a ResNet 18 architecture can train in under $\approx 2$m using an unoptimized PyTorch implementation on 1 GPU.
Furthermore, looking at the runtime behavior in \autoref{fig:runtime} we see that while allowing for more computation time increases the fidelity of the \method,
only a small number of training batches may be required to achieve a desirable level of statistical significance.

In scenarios where the deployed classifier is deemed high-risk (e.g.\ healthcare, justice system, education)
where each data point is a decision that affects a human being, we believe the additional computational expense is justified for an accurate and sensitive assessment of whether the classifier needs updating.
Having established the utility, accelerating the \method\;as well as building a deeper understanding of the runtime performance tradeoffs, is fertile ground for future work.

\begin{figure}[!htb]
    \vspace{-2mm}
    \centering
    \hspace*{2mm}\includegraphics[width=\linewidth]{images/runtime.pdf}
    \vspace{-7mm}
    \caption{\small \textbf{Runtime Characteristics:} We train 100 random runs of CDCs on 100 samples from CIFAR 10 and 10.1 and compute the \textit{disagreement satistic} as the difference $\psi\coloneqq \mathbb{E}[\phi_\boldQ - \phi_\boldP]$.
    While we see that while $\psi$ peaks near 50 training batches, only 10 batches are required for the Detectron disagreement test to reach an area under the TPR vs FPR curve (AUROC) of nearly 1 (i.e., perfect discrimination).
    Training CDCs for too long eventually lowers $\psi$ as $\mathbb{E}[\phi_\boldQ] \approx \mathbb{E}[\phi_\boldP]\approx 1$ meaning CDCs eventually overfit to disagreeing on all of their data.}
    \label{fig:runtime}
\end{figure}


    \chapter{Conclusion, Limitations and Future Work}\label{ch:conclusion}
    Our work presents a practical application of PQ learning towards creating a method capable of detecting covariate shifts given a pre-existing classifier.
On both neural networks and random forests, we showcase the efficacy of our method in being sensitive enough to detect covariate shift using a small number of unlabelled examples across several real-world datasets.
We remark on several characteristics of our algorithm that represent potential directions for future work:

\emph{Beyond Classification: }Our work here focuses on the case of classification (since a large number of pre-existing benchmarks in the literature focus on the same).
However, we believe there is a viable extension of our work to regression models where constrained predictors are explicitly learned to maximize test error according to the existing metric, such as mean squared error.
We leave this exploration for future work.

\emph{Beyond Covariate Shifts: }While covariate shift is the only type of shift that can be discovered from unlabeled data without additional assumptions, we acknowledge that other types of shift, such as label and concept shift, are prevalent in the real world.
Building learning-based methods to identify these types of shifts is another direction for future work.

Finally, we wish to highlight that while auditing systems such as the \method\ show promise to ease concerns when using learning systems in high-risk domains, practitioners interfacing with these systems should not place blind trust in their outputs.

    \bibliography{bib}

    \appendix


    \chapter{Appendix}\label{ch:appendix}


    \section{PQ Learning \& Rejectron}\label{sec:rejectron}
    We provide a summary of the original Rejectron algorithm for PQ learning~\citep{pqlearn} as it is the primary motivation for our work.
Rejectron (\autoref{alg:rejectron}) takes as input a labeled training set of $n$ samples $\mathbf{x}$ (iid over $\mathcal{P}$), an unlabeled test set of $n$ samples $\tilde{\mathbf{x}}$ (iid over $\mathcal{Q}$), an error $\epsilon$ and a weight $\Lambda$.
The output is a selective classifier, that predicts according to a base classifier $h$ if the input $x$ is inside some set $S$ and otherwise rejects (abstains from predicting).
\begin{equation}
    \left. h\right|_{S}(x) \coloneqq \begin{cases}
                                         h(x) & x\in S \\
                                         \text{reject} & x\not\in S
    \end{cases}
    \label{eq:selective_classifier}
\end{equation}
Under several assumptions and a special value $\epsilon^\star$, this selective classifier is guaranteed with high probability to have error less then $2 \epsilon^\star$ on $\tilde{\mathbf{x}}$
and a rejection rate below $\epsilon^\star$ on ${\mathbf{x}}$ (see Theorem 5.7 in~\citeauthor{pqlearn}).

\begin{algorithm}
    \SetKwComment{Comment}{$\ $\# }{ }
    \caption{Rejectron~\citep{pqlearn}}
    \label{alg:rejectron}
    \KwIn{train $\mathbf{x} \in X^{n}$, labels $\mathbf{y} \in Y^{n}$, test $\tilde{\mathbf{x}} \in X^{n}$, error $\epsilon \in[0,1]$, weight $\Lambda=n+1$}
    \KwOut{selective classifier $\left. h\right|_{S}$}
    $h\gets \operatorname{ERM}(\mathbf{x}, \mathbf{y})$ \Comment{assume black box oracle ERM to minimize errors}
    \For{$t=1,2,3, \ldots$}{
        1. $S_{t}\coloneqq \left\{x \in X: h(x)=c_{1}(x)=\ldots=c_{t-1}(x)\right\}$ \Comment{So $S_{1}=X$}
        2. Choose $c_{t} \in C$ to maximize $s_{t}(c)\coloneqq\operatorname{err}_{\tilde{\mathbf{x}}}\left(\left.h\right|_{S_{t}}, c\right)-\Lambda \cdot \operatorname{err}_{\mathbf{x}}(h, c)$ over $c \in C$\\
        3. If $s_{t}\left(c_{t}\right) \leq \epsilon$, then stop and return $\left.h\right|_{S_{t}}$
    }
\end{algorithm}
Rejectron starts by querying an empirical risk minimization (ERM) oracle that uses a 0--1 risk score over a concept class $C$ for a model $h$ that perfectly learns the training dataset.
A primary assumption for Rejectron to output a perfect model as well as for it to eventually find a selective classifier that meets the $\epsilon^\star$ bound is that the true decision function
(i.e., the function that creates the training labels) is also a member of $C$.
The authors refer to this setting as \textit{realizable}.

On the first iteration of the algorithm, Rejectron finds another model $c_1\in C$ that jointly maximizes the error with respect to $h$ on $\tilde{\mathbf{x}}$ while minimizing it on ${\mathbf{x}}$.
The authors show that they can efficiently solve this optimization problem using a single ERM query on a dataset of $n^2+n$ samples (see Lemma 5.1 in~\citeauthor{pqlearn}).

In every subsequent step $t>1$ a set $S_t$ is created where all models $h$ through to $c_{t-1}$ agree.
Another model $c_t\in C$ is found that maximizes the same objective as above but only on the intersection of $\tilde{\mathbf{x}}$ and $S_t$.
Upon termination a selective classifier $\left. h\right|_{S_t}$ is output.


    \section{Proofs}\label{sec:proofs}
    Below are the full proofs for Theorem 1 and 2 from \autoref{sec:from-constrained-disagreement-to-detecting-shift-with-hypothesis-tests}.

\begin{manualtheorem}{1}[Disagreement implies covariate shift]
  Let $f$ be a classifier trained on dataset $\boldP$ drawn from $\Pc$ and their corresponding labels.
    Let $g$ be a classifier that is observed to agree (classify identically) with $f$ on $\boldP$ and disagree on a dataset $\boldQ$ drawn from $\Qc$.
    If the rate which $g$ disagrees with $f$ on $n$ unseen samples from $\Qc$ is greater than that from $n$ unseen samples from $\Pc$ with probability greater than $p^\star \coloneqq \frac{1}{2}\left(1- 4^{-n} \binom{2n}{n}\right)$ there must be covariate shift.
\end{manualtheorem}
\begin{proof}
    Let $R_P=\emptyset^{P}_1+ \ldots+ \emptyset^P_n$ where each $\emptyset^P_i$ is an i.i.d Bernoulli random variable that describes the probability of $g$ disagreeing with $f$ on an unseen sample from $\Pc$.
    Additionally, let $R_Q=\emptyset^Q_1+ \ldots+ \emptyset^Q_n$ be defined similarly for $\Qc$.
    If $\Pc$ and $\Qc$ are equal then $\emptyset^P_i$ and $\emptyset^Q_i$ are equal by definition and the probability of observing $R_Q >R_P$ is tightly bounded by \autoref{eq:bound}.
    This is the tightest upper bound that is not a function of $\mathbb{E}[\emptyset^P_i]$, the proof can be found in Lemma \autoref{lemma:bound}.

    \begin{equation}
        \mathcal{P=Q} \implies \mathbb{P}\left(R_Q > R_P \right) \leq \frac{1}{2} \left(1-4^{-n} \binom{2 n}{n}\right) = \frac{1}{2} - O\left(\frac{1}{\sqrt{n}}\right)
        \label{eq:bound}
    \end{equation}


    The more helpful contrapositive statement says that if it is sufficiently likely that $R_Q>R_P$, then covariate distributions $\Pc$ and $\Qc$ must not be equal.

    \begin{equation}
        \mathbb{P}\left(R_Q > R_P \right) > \frac{1}{2} \left(1-4^{-n} \binom{2 n}{n}\right) \implies \mathcal{P\neq Q}
        \label{eq:th1result}
    \end{equation}

    This result naturally lends itself to identifying $\mathcal{P\neq Q}$ by rejecting an exact statistical hypothesis that $R_Q=R_P$ in favor of the alternative $R_Q>R_P$.
\end{proof}

\begin{lemma}
    \label{lemma:bound}
    Let $X$ and $Y$ be iid binomial random variables with distribution $\text{Bin}(n,p)$ then for all $n\in \mathbb{Z}\geq 0$:
    \begin{equation}
        P(X>Y) \leq \frac{1}{2}\left(1-4^{-n} \binom{2n}{n} \right) < \frac{1}{2}
        \label{eq:boundresult}
    \end{equation}
    Furthermore, \cref{eq:boundresult} is the tightest possible bound that does not depend on $p$.
\end{lemma}
\begin{proof}
    Let $Z$ be the distribution $X-Y$, while $Z$ itself is intractable to write down for arbitrary $n$, the characteristic function takes a convenient form
    \begin{align}
        \phi_Z(t;p) &= \mathbb{E}\left[e^{i t (x - y)}\right] = \mathbb{E}\left[e^{i t x}\right] \mathbb{E}\left[e^{-i t y}\right]\\
        &=\left(1+p \left(-1+e^{i t}\right)\right)^n \left(1+p \left(-1+e^{-i t}\right)\right)^n\\
        &=\left(-p^2 e^{-i t}-p^2 e^{i t}+2 p^2+p e^{-i t}+p e^{i t}-2 p+1\right)^n\\
        &=(1-2 p+p \cos (t)-i p \sin (t)+p \cos (t)+i p \sin (t)\\
        &\quad +2 p^2-p^2 \cos (t)+i p^2 \sin (t)-p^2\cos (t)-i p^2 \sin (t))^n \nonumber \\
        &=\left(p^2 (2-2 \cos (t))+p (2 \cos (t)-2)+1\right)^n
    \end{align}
    Since $X$ and $Y$ are identically distributed $P(Z>0) = P(Z<0)$ and so
    \begin{equation}
        P(Z>0) = \frac{1}{2}\left(1 - P(Z=0)\right)
        \label{eq:pz}
    \end{equation}
    \Cref{eq:pz} suggests that a tight upper bound of the form $P(Z=0)\geq \alpha$ implies a tight lower bound in the form $P(Z> 0) \leq (1-\alpha)/2$.
    To bound $P(Z=0)$ we first write it as an integral expression using the characteristic inversion formula for discrete random variables~\citep{ushakov_2011}
    \begin{equation}
        P(Z=0) = \frac{1}{2 \pi}\int_{-\pi}^{\ \pi} \phi_Z(t;p)d t
        \label{eq:prob0}
    \end{equation}
    Since $\phi_Z(t;p)$ has the form $(a(t) p^2 + b(t) p + 1)^n$ where $a$ and $b$ are real valued functions and $a(t) \geq 0\ \forall t \in \mathbb{R}$ (i.e an integer power of a quadratic equation with positive leading coefficient), then for any choice of $t$, $\phi_Z(t;p)$  will be globally minimized if and only if $p\to p^{\star}=-b(t)/\left(2 a(t)\right)$.
    For the particular form of $\phi_Z(t; p)$, $p^\star$ is simply $1/2$
    \begin{equation}
        p^\star = -\frac{b(t)}{2 a(t)} = -\frac{2 \cos (t)-2}{2 (2-2 \cos (t))} = \frac{1}{2}
        \label{eq:pstar}
    \end{equation}
    This result is intuitive as the variance of a binomial distribution $\text{Bin}(n,p)$ is maximized for any fixed choice of $n$ when $p=1/2$.
    We should note that \autoref{eq:pstar} appears problematic when $\cos(t)=1$, but in this case $\phi(t;p)$ becomes constant, hence $p$ cannot influence the upper bound.
    We can now write the upper bound for $P(Z=0)$
    \begin{align}
        P(Z=0) &= \frac{1}{2\pi}\int_{-\pi}^{\ \pi} \phi_{Z}(t;p) dt\\
        &\geq \frac{1}{2\pi}\int_{-\pi}^{\ \pi} \phi_Z\left(t;\frac{1}{2}\right) dt\\
        &=  \frac{1}{2\pi}\int_{-\pi}^{\ \pi} 2^{-n} (\cos (t)+1)^n dt \\
        &= 4^{-n}\binom{2n}{n} \label{eq:17}
    \end{align}
    The final expression in \cref{eq:17} can be found using the change of variables $z = e^{i t}$ and Cauchy's residue theorem~\citep{complex}
    \begin{align}
        I &= \ \ \frac{1}{2\pi}\int_{-\pi}^{\ \pi} 2^{-n} (\cos (t)+1)^n dt\\
        & = -\frac{i}{2 \pi } \oint_{|z|=1} 2^{-n} \frac{1}{z}\left(1+\frac{1+z^2}{2 z}\right)^n \, dz\ (\text{using } t\to -i \log z)\\
        &=  -\frac{i}{2 \pi } \oint_{|z|=1} 4^{-n} z^{-n-1} (z+1)^{2 n} dz\ (\text{simplifying}) \\
        & = \text{Res}\left( 4^{-n} z^{-n-1} (z+1)^{2 n}, 0\right)\ \ (\text{applying Cauchy's Theorem})\\
        & = \frac{1}{4^n n!} \lim_{z\to 0}\left( \frac{d^n}{d z^n} (z+1)^{2 n}\right)\\
        & = \frac{1}{4^n n!} 2n (2n -1) (2n-2)\ldots (n + 1)\\
        & = 4^{-n} \binom{2n}{n}
    \end{align}

    Combining \cref{eq:pz} with the bound from \cref{eq:17} we arrive at the conjectured upperbound
    \begin{align}
        P(X > Y) = P(Z>0) &= \frac{1}{2}\left(1 - P(Z=0)\right)\\
        &\leq \frac{1}{2}\left(1-4^{-n} \binom{2n}{n} \right) \text{ by \cref{eq:17}}\\
    \end{align}
    Finally we may use Sterling's approximation to show that $4^{-n} \binom{2n}{n} \in O\left(\frac{1}{\sqrt{n}}\right)$ and hence converges to $0$ as $n\to \infty$ leaving a limiting tight upper bound of $1/2$.
\end{proof}

\begin{manualtheorem}{2}[Probability of Disagreement: A Bayesian Perspective]
    \label{th:bayes-shift}
    Let $f$ be a classifier trained on dataset $\boldP$ drawn from the distribution $\Pc$ over $X$ and their corresponding labels.
    Let $g$ be a classifier observed to agree with $f$ on $\boldP$ but disagree on a dataset $\boldQ$ drawn from a distribution $\Qc$ over $X$.
    We denote the true probabilities that $g$ will disagree with $f$ on a sample from $\Pc$ and $\Qc$ as $p$ and $q$, respectively.
    Under a uniform prior $\mathcal{U}(0,1)$ for $p$ and $q$,
    if we observe that $g$ disagrees with $f$ on $m$ out of $M$ iid samples from $\mathcal{Q}$ while disagreeing with $n$ out of $N$ iid samples from $\mathcal{P}$, then
    the posterior probability that $g$ is truly more likely to disagree with $f$ on $\Qc$ compared to $\Pc$:
    \begin{align}
        \mathbb{P}[q > p] = &1 -\frac{(M+1)! (N+1)! (m+n+1)!}{(m+1)! n! (M-m)! (m+N+2)!} \times \label{eq:hyper}\\
        &\quad\quad\quad  _3F_2(m+1,m-M,m+n+2;m+2,m+N+3;1) \nonumber
    \end{align}
    Where $_p{F}_q$ is the generalized hyper-geometric function, implemented in several standard mathematical libraries.
\end{manualtheorem}
\begin{proof}
    For simplicity we consider the function $\text{dis}: X\to \{0,1\}$ that outputs $0$ if $f(x)=g(x)$ else $1$.
    We define the true disagreement rates $\textbf{p}$ and $\textbf{q}$ as
    \begin{equation}
        \mathbf{p} \coloneqq \mathbb{E}_{x\sim \mathcal{P}}[\text{dis}(x)]\text{ and } \mathbf{q} \coloneqq \mathbb{E}_{x\sim \mathcal{Q}}[\text{dis}(x)]
        \label{eq:defspq}
    \end{equation}
    Without any \textit{a-priori} knowledge of $\text{dis}$ we define the random variables $p$ and $q$ under uniform prior (i.e $p,q\overset{\text{i.i.d}}{\sim}\mathcal{U}(0,1)$) to encode our belief over the true values $\mathbf{p}$ and $\mathbf{q}$.
    Now we draw $N$ samples as $\mathbf{x} \overset{\text{i.i.d}}{\sim} \mathcal{P}^N$ and $M$ as $\tilde{\mathbf{x}} \overset{\text{i.i.d}}{\sim} \mathcal{Q}^M$ and compute the number of times $\text{dis}(x)$ equals $1$ on each set.
    \begin{equation}
    {n}
        \coloneqq \sum_{x\in \mathbf{x} } \text{dis}(x) \text{ and } {m}\coloneqq \sum_{x\in \tilde{\mathbf{x}} } \text{dis}(x)
        \label{eq:bayes_result}
    \end{equation}
    By definition in \autoref{eq:defspq} we know that $n$ and $m$ are draws from binomial distributions: $\mathcal{N}\sim \text{Bin}(N,p)$ and $\mathcal{M} \sim \text{Bin}(M,q)$ respectively.
    We can then compute the posterior probability density functions of $p$ and $q$ conditioned on the observations $\mathcal{N}=n$ and $\mathcal{M}=m$ using exact Bayesian inference.
    \begin{align}
        f_{p| {n}}(x)&\coloneqq \mathbb{P}[p = x | \mathcal{N}={n}] \\
        &= \mathbb{P}[\mathcal{N} = {n} | p = x] \underbrace{\mathbb{P}[p = x]}_{=1} \left(\int_{0}^1 \mathbb{P}[\mathcal{N}={n}|p=x] dx\right)^{-1}\\
        &={\binom{N}{n}} x^{n} (1-x)^{N-{n}}  \left(\int_0^1 {\binom{N}{n}} x^{n} (1-x)^{N-{n}} dx \right)^{-1}\label{eq:binointegral} \\
        &= x^{n} (1-x)^{N-{n}}  \left(\underbrace{B_x(n+1,-n+N+1)}_{\text{incomplete beta function}}\left. \right\rvert_{x=0}^{x=1}\right)^{-1} \\
        &=x^{n} (1-x)^{N-{n}}\left(\underbrace{\frac{n! (N-n)!}{(N+1) N!}}_{x=1} - \underbrace{0}_{x=0}\right)^{-1}\\
        &= \binom{N}{n} x^{n} (1-x)^{N-{n}} (N+1)
    \end{align}
    The integration in \autoref{eq:binointegral} is solved using the definition of the incomplete beta function.

    Without loss of generality we may also find $f_{q| m}$
    \begin{equation}
        f_{q| m}(x) = (M+1) \binom{M}{m} x^{m} (1-x)^{M-{m}}
        \label{eq:cond_pmf}
    \end{equation}
    Given these closed form posterior distributions for $p$ and $q$ we may compute the probability that the true value of $q$ is greater then $p$
    \begin{align}
        &\mathbb{P}[q > p | \mathcal{N}=n, \mathcal{M}=m] = \int_{y>x} f_{q|m}(y)f_{p|n}(x) dy dx\\
        &= \int_0^1\int_{x}^1 f_{q|m}(y)f_{p|n}(x) dy dx\\
        & = \binom{M}{m} \binom{N}{n}  (1+M) (1+N) \int_0^1 (1-x)^{-n+N} x^n \int_x^1 (1-y)^{-m+M} y^m dydx
    \end{align}
    This integral, while daunting, can easily be solved in closed form using the free online Wolfram Mathematica cloud (\href{https://www.wolframcloud.com/obj/87815d61-488b-4741-b62e-398aca5d8dae}{result link}).
    The solution is exactly \autoref{eq:hyper}.
    \begin{align}
        &\mathbb{P}[q > p | \mathcal{N}=n, \mathcal{M}=m] =1 -\frac{(M+1)! (N+1)! (m+n+1)!}{(m+1)! n! (M-m)! (m+N+2)!} \times \label{eq:th2-res}\\
        &  \quad\quad_3F_2(m+1,m-M,m+n+2;m+2,m+N+3;1) \nonumber
    \end{align}
    To gain intuition, a graphical representation of \autoref{eq:th2-res} is provided in \autoref{fig:bayes_dis}.
    From a practical standpoint, if a practitioner trains two classifiers $f$ and $g$ that appear to disagree more often on a new dataset than a baseline rate computed on an in-domain test set, they can decide to act on that observation.
    (e.g., collected more training data) at a particular belief threshold (e.g., probability greater than 80\%).
    Furthermore, there is a natural link between \autoref{eq:th2-res} and covariate shift as exact knowledge of $\mathbf{q} > \mathbf{p}$ by definition implies not only covariate shift $\Pc \neq \Qc$,
    but a type that is harmful by our original definition in TODO.
    Therefore, knowing the probability that $q>p$ is a useful measure of how likely, without any additional assumptions, that we are experiencing a harmful covariate shift.

    \begin{figure}
        \centering
        \includegraphics[width=\linewidth]{images/bayes_dis.pdf}
        \caption{Belief that the probability $q$ that two classifiers disagree on samples from $\Qc$ is greater then the probability $p$ that they disagree on $\Pc$ given an observation of $m$ disagreements out of $M$ samples from $\Qc$ and $n$ of $N$ disagreements on $\Pc$.
        We plot this probability for $M=20$ and $N=10000$. We observe that even for a small test set size, $M=20$, we can strongly believe that there is a true difference when $m/M$ is only slightly larger then $n/N$.}
        \label{fig:bayes_dis}
    \end{figure}

\end{proof}

    \section{Constrained Disagreement Loss Details}
    \label{sec:cons_dis_det}
    We elaborate on the technical details of the objective functions required for training constrained disagreement classifiers.

\subsection{Disagreement Cross Entropy}\label{subsec:disagreement-cross-entropy}
\xhdr{Intuition}
In our methodology, we propose the \textit{disagreement cross entropy} (DCE) as a simple loss function that encourages a classifier to disagree with a target label while otherwise outputting a high entropy prediction.
DCE is equivalent to simply flipping the target label and computing the regular cross-entropy in the binary classification case.

Consider a model that outputs a distribution $\{p_1, p_2, 1-p_1-p_2\}$ over $3$ classes, suppose we would like to train it to \textbf{not} output high probability for class $3$ i.e., minimize $p_3=1-p_1-p_2$.
An intuitive approach would be to \textit{maximize} the standard cross entropy loss $-\log(1-p_1-p_2)$ that one would equivalently \textit{minimize} if trying to output high probability for class $3$.
We show in \autoref{fig:dce} that this objective is unstable whereas the DCE $-\frac{1}{2}(\log(p_1) + \log(p_2))$ is convex, and has a bounded minimum corresponding to $p_1=p_2=1/2$.
In practice if one wishes a loss function with a minimum of zero it suffices to subtract $ \log(N-1)$ from \autoref{eq:anitcross}.
A visual description of the DCE is provided in \autoref{fig:dce}.
\begin{figure}[!htb]
    \centering
    \includegraphics[width=\linewidth]{images/dce.pdf}
    \caption{Consider a classifier outputs a distribution $\{p_1, p_2, 1-p_1-p_2\}$ over $3$ classes. We compare the optimization landscape induced by either (left) minimizing the negative cross entropy for the target $p_3=1-p_1-p_2$ (e.g trying not to predict class $3$) (center) minimizing our \textit{disagreement cross entropy} (DCE) for the same target
    and (right) minimizing the regular cross entropy (e.g trying to predict class $3$). We observe that naively minimizing the negative cross entropy results in an unbounded minimum, while the DCE is significantly more stable and scales similarly to the regular cross entropy. Gradients are overlaid to help better visualize the 3D geometry.}
    \label{fig:dce}
\end{figure}

\subsection{Validity of the DCE Measure}
We introduce a dentition of a \textit{a valid disagreement loss function} and show that the DCE loss satisfies it.
\begin{definition}[A Valid Disagreement Loss Function]
    Let $P$ be the set of all probability vectors of length $N$ whose maximum index is a fixed target label $y\in\{1,\ldots, N\}$ .
    Similarly, let $Q$ be the set of all probability vectors whose maximum index is \textbf{not} uniquely equal to $y$.
    For instance, in the three-dimensional case with $y=3$
    \begin{align*}
        &P=\{(p_1,p_2,1-p_1-p_2) \text{ s.t. } |\ {\max(p_1,p_2) < 1-p_1-p_2}\}\\
        &Q=\{(p_1,p_2,1-p_1-p_2) \text{ s.t. } |\ {\max(p_1,p_2) \geq 1-p_1-p_2}\}\\
        &\text{and } 0\leq p_1,p_2\leq 1\land p_1+p_2\leq 1
    \end{align*}
    A score $S(\mathbf{p}; y)$ is proper for disagreement if
    \begin{equation}
        \forall\ \mathbf{p}\in P\ \min_{\mathbf{q}\in Q} S(\mathbf{q}; y) \leq S(\mathbf{p}; y)
    \end{equation}
    which is to say that for every probability vector in $P$ there exists a probability vector in $Q$ that achieves a score at least as low.
\end{definition}

\begin{manualtheorem}{3}[Validity of DCE]
    The Disagreement Cross Entropy Loss is a valid disagreement loss function.
\end{manualtheorem}
\begin{proof}
    We show that the minimum DCE for a probability vector $\mathbf{q}\in Q$ is $\log(N-1)$ while the minimum DCE for $\mathbf{p}\in P$ is $\log(N)$.
    First note that by definition of the DCE in \autoref{eq:anitcross} the unique probability vector $\mathbf{v}$ that globally minimizes it with respect to a target class $y$ is
    the vector with elements $1/(N-1)$ for all indices $\mathbf{v}_i$ where $i\neq y$ and $\mathbf{v}_y=0$. $\mathbf{v}$ is a member of $Q$ because it does not have a unique maximum at position $y$.
    Computing the DCE of $\mathbf{v}$ with respect to the target $y$ gives:
    \begin{align*}
        \text{DCE}(\mathbf{v}, y) &= \frac{1}{1-N}\sum_{i=1}^N \log(\mathbf{v}_i) \mathbb{I}[i\neq y] \\
        &= \frac{1}{1-N} \log\left(\frac{1}{N-1}\right) \times (N-1)\\
        &=\log(N-1)
    \end{align*}
    Next note that since $\mathbf{p}_y$ does not contribute to minimizing the DCE which is in the form $\sum_{p\in \mathbf{p}\setminus \mathbf{p}_y} \log(p)$ the minimal solution must minimize the term $\mathbf{p}_y$. However to enforce $\mathbf{p}\in P$ we must have that $\forall\ i\in\{1,\ldots,N\}\setminus y,\ \mathbf{p}_y > \mathbf{p}_i$.
    Hence, minimizing $\mathbf{p}_y$ while making sure $\mathbf{p}\in P$ enforces $\mathbf{p}_y > 1/N$ for some $\epsilon > 0$ and all other $p_i < 1/N$ since if $\mathbf{p}_y \leq 1/N$ we could also chose some $p_i \geq 1/N$ and produce a vector $\mathbf{p}\in Q$. Without loss of generality choosing $\mathbf{p}_y = 1/N +\varepsilon$ for some small $\varepsilon > 0$ we must chose all other $\mathbf{p}_i= 1/N - \epsilon_i$ s.t. all $\epsilon_i > 0$ and $\sum_{i\in \{1,\ldots, N\}\setminus y} \epsilon_i = \varepsilon$. This gives a DCE of:
    \begin{align*}
        &\text{DCE}(\mathbf{p}, y) = \frac{1}{1-N}\sum_{i=1}^N \log(\mathbf{p}_i) \mathbb{I}[i\neq y] = \frac{1}{1-N} \sum_{i\in \{1,\ldots, N\}\setminus y} \log\left(\frac{1}{N}-\epsilon_i\right)\\
        &\text{ where } \lim_{\varepsilon\to 0 }\frac{1}{1-N} \sum_{i\in \{1,\ldots, N\}\setminus y} \log\left(\frac{1}{N}-\epsilon_i\right) = \log(N)
    \end{align*}
    For all vectors $\mathbf{p}\in P$ we cannot achieve a DCE of less than $\log(N)$ but the minimum DCE in $Q$ is $\log(N-1)$ which is less than $\log(N)$, hence proving that the DCE is a valid scoring rule for disagreement.
\end{proof}

\subsection{Implementation}
We provide a simple PyTorch style implementation for batched computation of the CDC objective in \autoref{eq:loss} given batch of logits and targets.
We assume \verb!logits! is a floating point vector of $(\text{batch\_size}\times\ N)$ logit values, \verb!targets! is a integer vector $(\text{batch\_size})$ of target classes, \verb!mask! is a 0-1 mask of $(\text{batch\_size})$ that is 1 at index $i$
if \verb!logits[i]! is $\boldP$ and 0 otherwise.
\begin{python}
    import torch, import torch.nn.functional as F
    def cdc_loss(logits, targets, mask, weight=1/(size of test set + 1)):
    # select data in p
    logits_p, targets_p = logits[mask], targets[mask]
    # select data in Q
    logits_q, targets_q = logits[1-mask], targets[1-mask]
    # cross entropy on P with targets
    loss_p = F.cross_entropy(logits_p, targets_p, reduction=None)
    # DCE on Q
    zero_hot = 1 - F.one_hot(targets_q, num_classes=N)
    loss_q = (logits_q * zero_hot).sum(dim=1) / (1 - N)
    + torch.logsumexp(logits_q)
    # Weighted mean
    return torch.cat([loss_p, weight * loss_q]).mean()
\end{python}

    \section{Experimental Details}
    \label{sec:expdet}
    \section{Experimental Details}
\label{sec:expdet}

\subsection{Base Classifiers}
\label{subsec:basclf}
For each dataset used in our experiments, we begin by training a \textit{base classifier} on the source domain portion of the dataset to use in subsequent experiments and baselines.
For a brief description of the datasets used and base classifiers, see \autoref{tab:datasets} and for a more detailed description of each dataset as well as what we have considered precisely as the source and shifted domains, see \autoref{subsec:predata}.

\xhdr{CIFAR 10}
We use a standard Resnet18 model pre-trained on ImageNet~\citep{deng2009imagenet} made available in the torchvision library~\citep{torchvision} (\verb!torchvision.models.resnet18(pretrained=True)!) although we reinitialize the last network layer to have an output size of $10$.
We use stochastic gradient descent (SGD) with a base learning rate of $0.1$, $L_2$ regularization of $5\times 10^{-4}$, momentum of $0.9$, a batch size of $128$ and a cosine annealing learning rate schedule with a maximum 200 iterations stepped once per epoch for a total of 200 epochs.
We use the standard CIFAR-10 training split normalized by its mean ($\mu=[0.4914, 0.4822, 0.4465]$) and standard deviation ($\sigma=[0.2023, 0.1994, 0.2010])$.
Every epoch, we randomly crop each image to a size of $32\times 32$ after applying a $0$ padding of four pixels to each spatial dimension, and we apply a horizontal flip with probability $0.5$.
This model achieves a test performance of $87\%$.
While this score is far from state-of-the-art on CIFAR-10, our goal is not to construct a perfect model.
We wish to create a \textit{reasonably good} model as an example of a model that could realistically be deployed in real-world settings.
When training deep ensembles, we only vary the random seed in the range $[0,\ldots ,4]$.

\xhdr{Camelyon 17}
We follow a similar approach to CIFAR 10.
However, we use two output features (for binary classification of cancerous or benign pathology), a batch size of $512$, the ADAM optimizer~\citep{DBLP:journals/corr/KingmaB14} with a base learning rate of $0.001$, $L_2$ regularization of $10^{-5}$ and a total of $5$ training epochs for which we select the model with the best validation accuracy.
This model achieves a test accuracy of $0.93$.
When training deep ensembles, we only vary the random seed in the range $[0,\dots,4]$.

\xhdr{UCI Heart Disease}
We train both neural networks and gradient boosted trees using the XGboost library~\citep{xgb}.
For the neural network model, we use a simple MLP with an input dimension of $9$, $3$ hidden layers of size $16$ with ReLU activation followed by a $30\%$ dropout layer and a linear layer to $2$ outputs (heart disease present or not).
We use $358$ samples for training and $120$ for validation.
We train for a maximum of $1000$ epochs and select the model with the highest AUC on the validation set, performing early stopping if the validation AUC has not increased in over 100 epochs.
This model achieves a test AUC computed on 119 samples of 0.85.
As with CIFAR 10 and Camelyon 17, we only vary the random seed in the range $[0,\dots,9]$ when training deep ensembles. Note that we chose a larger ensemble size here as models are fairly cheap to train.
Another important trick when using small $\boldQ$ sizes is to sample all of $\boldQ$ in each batch filling the best with a random set of samples from $\boldP$. Of procedure artificially inflates the size of $\boldQ$ so the hyperparameter $\lambda$ must account for this by picking up an extra multiplicative factor equal to $(\text{batches per epoch})^{-1}$.

When training gradient boosted trees using XGboost we employ standard library parameters ($\eta=0.1$, eval\_metric$=$auc, max\_depth$=6$, subsample$=0.8$, colsample\_bytree=$0.8$, min\_child\_weight$=1$, objective=binary:logistic, num\_round$=10$).
This model while taking less then $5s$ to train achieves a test AUC of 0.88.

\subsection{Constrained Disagreement Classifiers}
\label{subsec:constr}
We expand on the experimental details for learning constrained disagreement classifiers (\autoref{algo:constrained}).
When training a CDC $g_{(f,\boldP, \boldQ)}$ we start by creating a new dataset that combines all elements of the labeled set $\boldP$ and the unlabeled set $\boldQ$ with pseudo labels inferred by the base classifier $f$.
We store a single bit for each sample in the combined dataset to indicate if a sample was originally drawn from $\boldP$ or $\boldQ$.
When training CDCs with neural networks, we use the DCE loss ($\autoref{eq:loss}$)
under similar semantics as the pseudo-code implementation provided above.
When training discrete models, we resort to our generalized approach in \autoref{subsec:extension}.
To reduce training time we initialize $g$ using the exact same architecture/weights
as $f$ and apply the exact same optimization algorithm/learning rate used to train $f$ (see \autoref{subsec:basclf}).
For CIFAR 10, we train each CDC for a maximum of 10 epochs performing early stopping if the model drops in in-distribution validation performance
by over 5\%.

We enforce the early stopping criteria to help prevent CDCs from overfitting to the disagreement loss when the target dataset has not come from a harmfully shifted domain.
The intuition is the following: under the null, if a target dataset $\boldQ$ comes from the same distribution as a training dataset $\boldP$, then learning to disagree with $f$ on $\boldQ$ while constrained to agree on all of $\boldP$ can only be solved by overfitting to predict with high entropy on the specific examples in $\boldQ$, versus learning a distinct pattern that distinguishes the distributions.
Forcing a model to predict with high entropy on a subset of in-distribution datapoints can only hurt its associated in-distribution generalization, a phenomenon which we can directly assess by measuring validation performance.

The details for training CDCs on Camelyon 17 are the same as those described for CIFAR 10, however due to the large training set size ($302436$ samples) we simply select a random subset of size $50,000$ as $\boldP$ at each epoch -- a number we experimentally deemed as sufficient to achieve low in-distribution generalization error.
When training CDCs on the UCI Heart Disease dataset, we use XGBoost~\citep{xgb} with the same hyperparameters described in \autoref{subsec:basclf}.

For the runtime experiment presented in \autoref{fig:runtime} we train each CDC for only one batch, where each batch contains a set of 100 samples $\boldQ$ and is filled up to a batch size of 512 with random samples from $\boldP_{\text{train}}$. After every batch we eliminate all samples where the CDC disagrees with the base predictions.
We continue this for a maximum of 150 batches, but perform early stopping if 10 batches pass without at least one sample getting disagreed on.

\subsection{Description of Baseline Methods}
\label{subsec:baselines}
We compare the \method\ against several methods for OOD detection, uncertainty estimation and covariate shift detection found in recent literature.
\begin{enumerate}
    \item \textit{Deep Ensembles} shown by~\cite{trustuncert} to provide the most accurate estimates of predictive confidence under covariate shift.
    To compare directly with \method\ we test both the disagreements rates and the entropy distributions of the ensemble. See \autoref{sec:hyp} for more information on how these tests are run.
    \item \textit{Black Box Shift Detection (BBSD)}~\citep{bbsd} is overall best method across numerous synthetic benchmarks for covariate shift detection evaluated by~\citeauthor{failloud}.
    We follow the same evaluation and perform a univariate KS test on each dimension of the softmax output of the base classifier between $\boldQ$ and a held out set from the training distribution.
    Bonferroni correction is used to compute a single $p$-value as the minimum value divided by the number of tests.
    We guarantee significance using the same permutation approach described in \autoref{sec:hyp}.
    \item \textit{Relative Mahalanobis Distance (RMD)}~\citep{relmahala} (a method designed specifically for identifying near OOD samples) using the penultimate layer of a pretrained model.
    We test for covariate shift by performing a KS test directly on the distribution of RMD confidence scores derived on $\boldQ$ and $\Pun$.
    \item \textit{Classifier two sample test (CTST)}~\citep{paz2017revisiting}. Using the same architecture as and initialization as the base classifier we reconfigure the output layer and we train
    a domain classifier on half the test data with source data labeled as 0 and test data as 1. We then test this models accuracy on the other half of the test data and compare its performance to random chance using a binomial test (see \autoref{sec:hyp} for more details).
    While this method is technically sound it is not suitable for the low data regime where learning a domain classifier on half the test data is unlikely to generalize beyond random performance on the other half.
    \item \textit{Deep Kernel MMD}~\citep{liu2020learning}. We use the authors original source code available at \url{https://github.com/fengliu90/DK-for-TST} to perform the deep kernel MMD test.
    \item \textit{H-Divergence}~\citep{zhao2022comparing}. Most similar to our approach, this work proposes a two sample test based on the output of a learning model after training on either source or target data. Specifically, the authors fit a model to both the source dataset $\Pc$, the target dataset $\Qc$ and a uniform mixture $(\Pc+\Qc)/2$.
    Under the null hypothesis $\Pc=\Qc$ the loss in each case is equal in expectation. However when $\Pc \neq \Qc$, the generalized entropy of the mixture distribution may be be larger. In practice the authors fit three VAE~\citep{DBLP:journals/corr/KingmaW13} models and compute the test statistic $\ell((\Pc+\Qc)/2) -\min(\ell(\Qc), \ell(\Pc))$, where $\ell$ is the VAE loss
    computed as a sum of the binary cross entropy reconstruction loss and the KL divergence regularizer. The perform 100 runs where the null hypothesis (e.g. sample $\boldQ$ from $\Pc$) and one where it does not. Significance is determined in the standard way be observing if the true test statistic exceeds the $95^\text{th}$ percentile of the test statistic distribution under the null hypothesis.
    Unfortunately this method, while state of the art on several benchmarks including the MNIST vs Fake MNIST two sample test, demonstrated low utility on more complex tasks with smaller sample sizes. After a discussion with the authors, we attempted to improve the results by first pretraining the VAE to produce valid samples and reconstructions under the source distribution and computing the H-Divergence statistic
    after finetuning. Despite this effort, we still was low statistical significance with small sample sizes likely due to the noisy nature of training VAE's in the low data regime. We use the authors original source code available here \url{https://github.com/a7b23/H-Divergence}.
\end{enumerate}


\section{Datasets}
\label{sec:data}

\subsection{Sources and Licensees}\label{subsec:sources-and-licensees}
\begin{itemize}
    \item \href{https://peltarion.com/knowledge-center/documentation/terms/dataset-licenses/cifar-10}{CIFAR-10} (MIT License Copyright (c) 2013 Valay Shah)
    \item \href{https://camelyon17.grand-challenge.org/Data/}{Camelyon-17} (CC0 1.0 Universal Public Domain Dedication)
    \item \href{https://archive.ics.uci.edu/ml/datasets/Heart+Disease}{UCI Heart Disease} (Creative Commons Attribution 4.0 International)
\end{itemize}

\subsection{Prepossessing, Shift Descriptions and Model Performance}
\label{subsec:predata}
We provide full details on the three datasets used in our experiments, including any preprocessing steps and what splits we considered as source domain $\Pc$ and target domain $\Qc$.

\xhdr{CIFAR 10/10.1}
We use the well known CIFAR 10 dataset~\citep{krizhevsky2014cifar} as the source domain for training base classifiers (\autoref{subsec:basclf}) and the new CIFAR 10 test set CIFAR 10.1 containing 2000 class balanced images~\citep{cifar101} as a source of harmful distribution shift. Although the images in CIFAR 10.1 appear to be visually very similar to CIFAR 10 most classifiers trained on CIFAR 10 drop significantly in performance (3\% to 15\%~\citep{cifar101}) when tested on CIFAR 10.1.
\begin{figure}[!htb]
    \centering
    \includegraphics[width=\linewidth]{images/ciafr101.pdf}
    \caption{Cifar 10 vs Cifar 10.1. (Image borrowed from the technical report "Do CIFAR-10 Classifiers Generalize to CIFAR-10?"~\citep{ciafr10101})}
    \label{fig:cifar101}
\end{figure}

\xhdr{Camelyon 17}
As described by the original authors~\citep{camelyon} the Camelyon benchmark is a new and challenging image classification dataset consisting of 327,680 color images ($96 \times 96$) extracted from histopathologic scans of lymph node sections.
Each image is annotated with a binary label indicating the presence of metastatic tissue in the center $32\times 32$ pixel region.
In our experiments, we use the WILDS~\citep{wilds} framework to facilitate download, preprocessing, as well as source/target splits for Camelyon.
As shown in \autoref{fig:camelyon}, the source domain is chosen as data from hospitals $1,2,3$.
In contrast, the test domain is collected from hospital $5$, which visually shows significantly higher contrast due to different data acquisition equipment/methods.
\begin{figure}[!htb]
    \centering
    \hspace{-1cm}
    \includegraphics[width=\linewidth]{images/camelyon.pdf}
    \caption{Samples images from the Camelyon 17 dataset~\citep{camelyon}. Using the standard set by the WILDS framework~\citep{wilds} we use hospitals 1-3 as the source domain for training and validating models, and hospital $5$ as the target domain for assessing distribution shift.}
    \label{fig:camelyon}
\end{figure}

\xhdr{UCI Heart Disease}
The UCI Heart Disease (UCI-HD) dataset~\citep{misc_heart_disease_45} consists of 76 attributes collected from four unique patient databases in Cleveland, Hungary, Switzerland, and the VA Long Beach.
We select nine features out of the commonly used 14 to minimize the portion of missing values.
These features are \{age, sex, chest pain type, resting blood pressure, serum cholesterol, fasting blood sugar, resting electrocardiographic results, maximum heart rate achieved, exercise-induced angina\}.
The prediction task is to determine the diagnosis of heart disease (also known as angiographic disease status), which is given in a range from 0-4, where 0 indicates healthy and 1-4 indicates a severity level based on the narrowing of major blood vessels.
Following prior work~\citep{Chaki2015ACO} we only consider the simplified binary classification task for differentiating patients with a normal angiographic status (label of 0) from those with abnormal status (label $>0$).
We select the source domain as the Cleveland and Hungary databases and the target domain as the Switzerland and VA Long Beach databases.
A graphical overview of the marginal feature distributions for the source and target domain is shown in \autoref{fig:uci}.

\begin{figure}
    \centering
    \includegraphics[width=0.85\linewidth]{images/uci.pdf}
    \caption{Marginal distributions for each of the nine variables chosen from the UCI Heart Disease dataset for our experiments. The source domain (yellow) is chosen from the Cleveland and Hungary patient databases,
        while the shifted target domain (blue) is selected from the Swiss and Long Beach databases. Although the distributions are visually similar, a simple neural network classifier that achieves an AUC of 0.85 on the source domain drops to 0.42 on the target domain.}
    \label{fig:uci}
\end{figure}

To allow for out-of-the-box training of deep neural networks on the UCI HD dataset, we use the missing value synthesis functional in Wolfram Mathematica~\citep{Mathematica}.
The algorithm uses density estimation and mode finding on conditioned distributions to synthesize missing values.
See the language guide page titled \href{https://www.wolfram.com/language/12/high-level-machine-learning/synthesize-missing-values-in-numeric-data.html?product=language}{Synthesize Missing Values in Numeric Data} for a more detailed description.
To aid in future research, we provide a copy of our processed dataset in \href{https://anonymous.4open.science/r/detectron-D7BA/data/uci_heart_torch.pt}{\texttt{<our github repo>/data/uci\_heart.pt}}.

A summary of the three datasets used as well as the description of shifts and effects on model performance is provided in \autoref{tab:datasets}.
\begin{table}[!htb]
    \centering
    \setlength\tabcolsep{1.5pt}
    \renewcommand{\arraystretch}{0.5}
    \small
    \caption{\small \textbf{Datasets:} We investigate three different forms of covariate shift. To verify that these shifts are indeed harmful to the models, we report performance in both the shifted and unshifted domains.
    Examples and further descriptions of unshifted/shifted splits of each dataset are given in \autoref{sec:data}.}
    \vspace{.5cm}
    \resizebox{.9\textwidth}{!}{
        \begin{tabular}{cccccc}
            \toprule[1.5pt]
            Domain/Task & {Dataset} & \thead{Shift} & \thead{Metric} & \thead{(Unshifted)} & \thead{(Shifted)} \\\midrule[1.5pt]
            \thead{Natural Images \\\emph{Object classification}} & \thead{CIFAR-10/10.1 \\~\citep{cifar101}} & \thead{Data Collection \\ Process} & \thead{Accuracy} & \thead{0.87 (Resent18)} & \thead{0.77 (Resent18)} \\\midrule
            \thead{Histopathological Images \\ \emph{Metastases Detection}} & \thead{Camelyon-17 \\\citep{camelyon}} & \thead{Different \\Hospitals} & \thead{Accuracy} & \thead{0.93 (Resent18)} & \thead{0.81 (Resent18)} \\\midrule
            \thead{Tabular Medical Data \\\emph{Angiographic Status}} & \thead{UCI Heart Disease \\\citep{misc_heart_disease_45}} & \thead{Different \\Countries} & \thead{AUROC} & \thead{0.88 (xgboost)\\0.85 (MLP)} & \thead{0.70 (xgboost)\\0.42 (MLP)} \\\midrule
        \end{tabular}}
    \label{tab:datasets}
    \vspace{-3mm}
\end{table}


\end{document}